\chapter{Software portability}\label{porting}

Ideally, any software would be usable on any operating system, platform, and any processor architecture.
Existence of term "porting", derived from the Latin portāre which means "to~carry", proves that this ideal situation does not occurs that often, and acctual process of "carrying" software to system with different environment is most probably required.
Porting is process not documented and required if we want to run thing on another platform.
Porting is also used to describe procces of converting computer games to became platform independent.

Software porting procces might be hard to distinguish with building software.
The reason might be that in many cases building software on desired platform is enough, when software application does not work "out of the box".
This kind of behavior, an application that works immediately after or even without any special installation and without need for any configuration or modification, is ideal.
It often happens when we copy application from one computer to another, not realizing that they have processors with same instructions set and same or very simmilar operating system therefore envoroment.
But let us be realistic, it depends on many things, such as processor architecture to which was application written (compiled), quality of design, how application is meant to be deployed and of course application's source code.

Number of significantly different central processor units (CPUs), and operating systems used on the desktop or server is much smaller than in the past.
Embeded ARM MIPS  RISC CISC x8664.

Nowadays, the goal should be to develop software which is portable between preferred computer platforms (Linux, UNIX, Apple, Microsoft).
If software is considered as not portable, it could not have to mean immediately that it is not possible, just that time and resources spent porting already written software are almost comparable, or even significantly higher than writing software as a whole from scratch.
Effort spent porting some software product to work on desired platform must be little, such as copying already installed files to usb flash drive and run it on another computer.
This kind of approach might most probably fail, due to not present dependencies of third party libraries not present on destination computer.
Despite dominance of the x86 architecture there is usually a need to recompile software running, not only on different operating systems, to make sure we have all the dependencies present.

To simplify portability, even on distinguished processors with distant instruction sets, modern compilers translate source code to a machine-independent intermediate code.
But still, in the embedded system market, where OpenWrt belongs to, porting remains a significant issue \cite{porting_software}.

\section{OpenWrt's toolchain}

Embeded devices are not meant for building on them because they does not have enough memory nor computation resorces as ordinary personal computers do(see Table \ref{routerspec}).
Building on such device would be time consuming and may result to overheating, which could cause hardware to fail.
For this particular reason package building is done with cross-compiler.

Cross-compilation is done by set of tools called toolchain and it consists of:

\begin{itemize}
    \item compiler
    \item linker
    \item a C standard library
\end{itemize}

For porting to "target"  system (OpenWrt) this toolchain has to be generated on "host" system.
The toolchain can be achieved in many different ways.
The easiest way is undoubtedly to find a .rpm (or .deb) package and have it installed on our "host" system.
If a binary package with desired toolchain is not available for your system or is not available at all, there might be need to compile a custom toolchain from scratch.\footnote{tools like crosstool-ng (https://github.com/crosstool-ng/crosstool-ng) may help}

In case of OpenWrt we have available a set of Makefiles and patches called buildroot which is capable of generating toolchain.

\subsection{Compiler}

In a nutshell compilers translate the high level programming language source code into lower level language such as machine understandable assembler instructions.
To do so compiler is performing many operations starting with preprocessing.

Preprocessing is

Lexical analysis,

Parsing,

Semantic analysis (syntax-directed translation),

Conversion of input programs to an intermediate representation,

Code optimization

Code generation

They can create binary executable files and object files.

Cross-compiler is programming tool capable of creating executable code that is supposed to run on a "target" architecture, in a similar or completely different enviroment, while working on a different "host" architecture.
Reason for using cross-compiling might be to separate the build environment from target environment as well.
OpenWrt toolchain uses gcc compiler, and it is one of the most important parts of toolchain.

\subsection{Linker}

bla ble bli

\subsection{C standard library}

The OpenWrt's cross-compilation toolchain uses musl, a tiny C standard library.

\section{OpenWrt's buildroot}

OpenWrt's buildroot is a build system capable of generating toolchain, and also a root filesystem (also called sysroot), enviroment tightly bound to target.
Build system can be configured for any device that is supported by OpenWrt.

The root filesystem in general is a mere copy of the file system of target's platform.
In many cases just having the folders /usr and /lib would be sufficient, therefore we do not need to copy nor create the entire target file system on our host .

It is a good idea to store all these things, the toolchain and the root filesystem in a single place.
With using OpenWrt's buildroot we will have this covered.
Be tidy and pedantic, because cross-compiling can easily become a painful mess!

\subsection{Getting buildroot}

There are many mirrors containg openwrt SDK buildroots but when developing upstream projet upstream build root might be best solution.
problems with SDKs on mirrors are that they are most likely outdated and dependencies required to make it possible to build on this SDK might be not possible to get on your platform
we can clone buildroot from upstream with git Worflow should be to fork, clone forked copy to ourselves

\subsection{Buildroot prerequisites}

Let us demonstrate what are minimum requirements of space and size of RAM (Random Access Memory) for building packages for Openwrt using its buildroot.
For generating an installable OpenWrt firmware image file with a size of e.g. 8MB, we will need at least:

\begin{itemize}
\item ca. 200 MB for OpenWrt build system
\item ca. 300 MB for OpenWrt build system with OpenWrt Feeds
\item ca. 2.1 GB for source packages downloaded during build from the Feeds
\item ca. 3-4 GB to build (i.e. cross-compile) OpenWrt and generate the firmware file
\item ca. 1-4 GB of RAM to build Openwrt.(build x86's img need 4GB RAM)
\end{itemize}

These are specifications of embedded device TL-WR842Nv3 a regular wireless router manufactured by TP-LINK which was used to test all packages related to Tang server porting effort:

\begin{table}[h]
\centering
\label{routerspec}
\begin{tabular}{c|c}
\hline
Model           &   TL-WR842N(EU)                   \\ \hline
Version         &   v3                              \\ \hline
Architecture:   &   MIPS 24Kc                       \\ \hline
Manufacturer:   &   Qualcomm Atheros                \\ \hline
Bootloader:     &   U-Boot                          \\ \hline
System-On-Chip: &   Qualcomm Atheros QCA9531-BL3A   \\ \hline
CPU Speed:      &   650 MHz                         \\ \hline
Flash chip:     &   Winbond 25Q128CS16              \\ \hline
Flash size:     &   16 MiB                          \\ \hline
RAM chip:       &   Zentel A3R12E40CBF-8E           \\ \hline
RAM size:       &   64 MiB                          \\ \hline
Wireless:       &   Qualcomm Atheros QCA9531        \\ \hline
Antenae(s):     &   2 non-removable                 \\ \hline
Ethernet:       &   4 LAN, 1 WAN 10/100             \\ \hline
USB:            &   1 x 2.0                         \\ \hline
Serial:         &   ?                               \\ \hline
\end{tabular}
\caption{TL-WR842Nv3 specifications}
\end{table}

The actual sum of space required only for buildroot to work correctly which is about 6.4 GB compared to available storage space on wireless router should be demonstrative why is building for embedded devices done with cross-compiling.
The another reasons, not to just compare internal storage which might be extended (as shown on apendix TODOapendix), are device's minimalistic RAM size and not that "powerfull" CPU.\todo{add more about CPU}

To be able to work with OpenWrt's bildroot we will need lots of buildroot dependencies.
But first we need to install all the dependencies that we need for this buildroot to be functional and ready to produce packages for our desired architecture on OpenWrt platform
This is how we install buildroot prerequisites on Fedora 27 system:

\begin{lstlisting}[language=bash,basicstyle=\ttfamily\footnotesize,label=fedora-install,caption=Prerequisites installation on Fedora 27]
    dnf install -y subversion binutils bzip2 gcc gcc-c++ \
    gawk gettext git-core flex ncurses-devel ncurses-compat-libs \
    zlib-devel zlib-static make patch unzip perl-ExtUtils-MakeMaker \
    perl-Thread-Queue  glibc glibc-devel glibc-static quilt sed \
    sdcc intltool sharutils bison wget openssl-devel
\end{lstlisting}

Please note that this command has to be run by user with root privileges.

we can clone packages from upstream with git Worflow should be to fork, clone forked copy to ourselves set upstream origin then when creating or wanting to edit any package we should create branch for this changes and then commit patches to this branch
when branch is ready create pull request and hope for the best that it wont be automatically revoked and just minor changes will be required.
after comunication with upstream and resolving all the possible issues our patch is ready to push to upstream.

where upstream changes are? how i can download ltest packages and what should i do to have them on my system?

\section{Working with buildroot}
\begin{enumerate}
    \item Do everything as non-root user!
    \item Issue all OpenWrt build system commands in the <buildroot> directory, \\e.g. ~/openwrt/trunk/
    \item Do not build in a directory that has spaces in its full path.
    \item Change ownership of the directory where you downloaded \\the OpenWrt to other than root user
\end{enumerate}

describe buildroot topology and descibe what every single thing is what for
make will trigger making all packages not like in sdk
difference working with sdk and buildroot that sdk build only packages and buildroot can buld only package but it is designed to build entire image
for specified architecture respectively device from list
do not know how to add there new device but may provide link
configured with make menuconfig shown here firstly select architecture and configuration will be generated into .config :
in sdk .config is pre generated and make menuconfig (need to find out if possible)

on sdk make will trigger only building packages in directory package and it has to be done right way
it is also possible to do this with build root but is better to have upstream or own packages there in right directory

screen

make menuconfig is to select all packages which we want to build  todo findout if it is neccessary to buld base packages for selected

{\tt \$ make}

{\tt \$ feeds install -a}

tm tmp is needed if package directory has been edited to regenerate metadata of what (todo find out)

it is better to know what we are installig with feeds install and to choose only required packages for one newly choosed package to build

\section{Makefiles for OpenWrt's packages}

This is very important part of creating new packages therefore porting them to openwrt this file is similar to specfiles on fedora and

\begin{lstlisting}[language=c,basicstyle=\ttfamily\footnotesize,label=c,caption=Basic C code.]
    #include<stdio.h>
    #include<iostream>
    int main(void)
    {
        printf("Chello World\n");
        // comment
        return 0;
    }
\end{lstlisting}
