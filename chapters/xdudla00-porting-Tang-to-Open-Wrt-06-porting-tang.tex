\chapter{Porting Tang}\label{porting-tang}

After successfull cross-compilation of jose we have all the dependencies "ready" and packaged except the systemd.
systemd is only one of the many implementations (inetd, launchd, ucspi-tcp, xinetd) of a super-server providing socket activation.



\section{Socket activation}\label{socket_activation}

Socket activation is a technology provided by a super-server (also called a service dispatcher daemon).
A super-server starts other servers when needed as well, normally with access to them checked by a TCP wrapper.
It uses very few resources when in idle state.

A service designed for the socket activation would behave as bare CLI application with input read from stdin (standard input) and output written to stdout (standard output).
Tang is exactly this kind of an application and because of that we need to configure socket activation\cite{super_server}.



\subsection{xinetd}
xinetd listens for incoming requests over a network and launches the appropriate service for that request.
Requests are made using port numbers as identifiers and xinetd usually launches another daemon to handle the request.
This is reflected on Figure \ref{fig_xinetd} xinetd socket activation below.
\begin{figure}[h]
    \centering
    \includegraphics[scale=0.9]{figures/xinetd.pdf}
    \caption{xinetd socket activation}
    \label{fig_xinetd}
\end{figure}
xinetd features access control mechanisms such as TCP Wrapper ACLs (access control lists), extensive logging capabilities, and the ability to make services available based on time.
It can place limits on the number of servers that the system can spawn.
xinetd is listening on behalf of the services.
Whenever a connection would come in an instance of the respective service will be spawned with using stdin and stdout of the service application\cite{xinetd}.



\section{Package the Tang}
Similarly to José we need to create a new package for OpenWrt.
Let us create a branch and directory for the Tang:
\begin{lstlisting}[columns=fixed,basicstyle=\ttfamily\footnotesize,tabsize=4,backgroundcolor=\color{yellow!10}]
$ git checkout master
$ git chechout -b add-tang
$ mkdir -p utils/tang/
\end{lstlisting}
The Tang project is owned by same owner on GitHub as José.
We should visit the project releases page\footnote{https://github.com/latchset/tang/releases/} and get the Tang version v6.
Then add following lines to the Makefile similarly as with José's Makefile:
\begin{lstlisting}[columns=fixed,basicstyle=\ttfamily\footnotesize,tabsize=4,backgroundcolor=\color{yellow!10}]
include $(TOPDIR)/rules.mk

PKG_NAME:=tang
PKG_VERSION:=6
PKG_RELEASE:=1

PKG_SOURCE:=$(PKG_NAME)-$(PKG_VERSION).tar.bz2
PKG_SOURCE_URL:=\
https://github.com/latchset/$(PKG_NAME)/releases/download/v$(PKG_VERSION)/

PKG_HASH:=1df78b48a52d2ca05656555cfe52bd4427c884f5a54a2c5e37a7b39da9e155e3


PKG_INSTALL:=1
PKG_BUILD_PARALLEL:=1

PKG_FIXUP:=autoreconf

include $(INCLUDE_DIR)/package.mk
\end{lstlisting}
Do not forget to add the package description which should have section dependencies filled.
The libhttp-parser dependency used for parsing HTTP requests.
José the library and tool for the Javascript Object Signing and Encryption.
xinetd is a runtime dependency, the actual build proccess of the Tang does not require its libraries but to have its socket activation available in runtime.
The bash dependency is there for a reason that Tang's tangd-update and tangd-keygen executables are bash scripts.
These scripts are complex and are using data structures that are not available for OpenWrt's default shell - ash.
\begin{lstlisting}[columns=fixed,basicstyle=\ttfamily\footnotesize,tabsize=4,backgroundcolor=\color{yellow!10}]
define Package/tang
  SECTION:=utils
  TITLE:=tang v$(PKG_VERSION) - daemon for binding data to a third party
  DEPENDS:=+libhttp-parser +xinetd +jose +bash
  URL:=https://github.com/latchset/tang
endef
\end{lstlisting}
The Tang package will be present in utils section of the
Let us add a brief description to our new package using description define:
\begin{lstlisting}[columns=fixed,basicstyle=\ttfamily\footnotesize,tabsize=4,backgroundcolor=\color{yellow!10}]
define Package/tang/description
	Tang is a small daemon for binding data to the presence of a third party
endef
\end{lstlisting}
The buildroot should know where to install the Tang's binaries.
Let us define a install section and use standard tangd binary location as on Fedora OS:
\begin{lstlisting}[columns=fixed,basicstyle=\ttfamily\footnotesize,tabsize=4,backgroundcolor=\color{yellow!10}]
define Package/tang/install
	$(INSTALL_DIR)	$(1)/usr/libexec
	$(INSTALL_BIN)	\
			$(PKG_INSTALL_DIR)/usr/lib/$(PKG_NAME)d*	$(1)/usr/libexec/
endef
\end{lstlisting}
Let us not forget the last line which allows the actual "magic" to happen:
\begin{lstlisting}[columns=fixed,basicstyle=\ttfamily\footnotesize,tabsize=4,backgroundcolor=\color{yellow!10}]
$(eval $(call BuildPackage,$(PKG_NAME)))
\end{lstlisting}
We can now merge these changes to feeds and try to build our freshly created Tang package:
\begin{lstlisting}[columns=fixed,basicstyle=\ttfamily\footnotesize,tabsize=4,backgroundcolor=\color{yellow!10}]
$ git commit -a
$ git push --set-upstream origin add-tang
$ git checkout new_packages
$ git merge add-tang
\end{lstlisting}
The new packagee tang will be available in the {\it Extra packages} section of the menuconfig after updating feeds:
\begin{lstlisting}[columns=fixed,basicstyle=\ttfamily\footnotesize,tabsize=4,backgroundcolor=\color{yellow!10}]
$ ./scripts/feeds update packages
$ ./scripts/feeds install jose
$ make menuconfig
$ make package/jose/{clean,compile}
\end{lstlisting}
After first try to build the tang package we will encounter the systemd dependency errror:
\begin{lstlisting}[columns=fixed,basicstyle=\ttfamily\footnotesize,tabsize=4,backgroundcolor=\color{yellow!10}]
configure: error: Package requirements (systemd) were not met:

No package 'systemd' found

Consider adjusting the PKG_CONFIG_PATH environment variable if you
installed software in a non-standard prefix.
\end{lstlisting}
We did not defined it for Makefile but the cross-compilation of the package will try to configure and compile downloaded sources.
Compiler will try to find the systemd dependency as it is defined in the "file" in Tang repository.
We shall remove this builtime dependency.

To do so we will remove a requirement for systemd from the Tang's configure.ac and Makefile.am file.
These patches are too extensive to be demonstrated.
{\it Makefile\_am.patch} and {\it configure\_ac.patch} can be found in submitted pull-request files\footnote{https://github.com/openwrt/packages/pull/5447/files} on GitHub.

To have sources patched before compilation we have to crate a directory for them in the package's feeds repository we forked on branch containing commit adding the Tang and copy them to created directory:
\begin{lstlisting}[columns=fixed,basicstyle=\ttfamily\footnotesize,tabsize=4,backgroundcolor=\color{yellow!10}]
$ cd packages-OpenWrt
$ git checkout add-tang
$ mkdir -p utils/tang/patches
\end{lstlisting}
Patches included in this directory are automatically applied on the sources downloaded from the mirror in the build time.

To have these changes in our feeds in buildroot commit them and push to add-tang branch.
After these changes pushed and merged with {\it new\_packages} branch a rebuild of the tang package we will succeed.
Now we have Tang package ready to be installed on our device.

We are avoiding troubles with using the libhttp-parser in version 2.8.0.
These self issued problems are described in following subsection \ref{porting_problems} Obstacles of delivering Tang.
The most important part after successfull build would be to configure it right way.



\subsection{Obstacles of delivering Tang package}\label{porting_problems}



\paragraph{}
