\chapter{Configuring the Tang on OpenWrt}\label{config}

Having the installable packages in our buildroot is only the half of the work done.
We need to install them on the actual device and setup the environment especially for the Tang server to work correctly.
The installation of the packages on OpenWrt is done with the opkg package manager.

The opkg (Open Package Management System) is a lightweight package manager used to download and install OpenWrt packages.
These packages could be stored somewhere on device's file-system or the package manager will download them from local package repositories or ones located on the Internet mirrors.
Users already familiar with GNU/Linux package managers like dnf, yum, apt/apt-get, pacman, emerge etc. will recognize the similarities.
It also has similarities with NSLU2's Optware, also made for embedded devices.

Opkg attempts to resolve dependencies with packages in the available repositories/mirrors.
If the opkg fails to find the dependency, it will report an error, and abort the installation of selected package.

By removing systemd lines from sources we end up not have automatic updates of the Tang's cache and xinetd's socket activation to set up.



\section{Install the Packages}

The packages that have been built in our buildroot are not available in any online mirror yet.
To make them available for OpenWrt we should create a pull-request tith the change and work with the community to accept it.
Before we do so, we have to make sure, that the built packages are working try installing them and testing their functionality.

To get our newly built packages to the target device running the OpenWrt we can upload them using scp to device's file-system.
After successful build, the packages are present in the bin/packages/mips\_24kc/packages/ directory (the location is bound to device configuration):
\begin{lstlisting}[columns=fixed,basicstyle=\ttfamily\footnotesize,tabsize=4,backgroundcolor=\color{yellow!10}]
$ scp bin/packages/mips_24kc/packages/*.ipk \
    root@192.168.0.1:/root/custom-packages/
\end{lstlisting}
This command will upload every package built before in builroot to the device's directory {\tt /root/custom-packages/}.
The buildroot should contain at least these files:
\begin{lstlisting}[columns=fixed,basicstyle=\ttfamily\footnotesize,tabsize=4,backgroundcolor=\color{yellow!10}]
$ ls bin/packages/mips_24kc/packages/
bash_4.4.12-1_mips_24kc.ipk
jansson_2.10-1_mips_24kc.ipk
jose_10-1_mips_24kc.ipk
libhttp-parser_2.8.0-1_mips_24kc.ipk
libjose_10-1_mips_24kc.ipk
Packages
Packages.gz
Packages.manifest
Packages.sig
tang_6-1_mips_24kc.ipk
xinetd_2.3.15-5_mips_24kc.ipk
\end{lstlisting}
After the successful upload, connect to the device and install packages.
We recommend installing only newly built or updated packages.
opkg will resolve known dependencies from the mirrors and will install them.
The {\tt /root/custom-pacakges} is not set up as custom opkg feed, thus we need to install tang and ist depencencies manually in order:
\begin{lstlisting}[columns=fixed,basicstyle=\ttfamily\footnotesize,tabsize=4,backgroundcolor=\color{yellow!10}]
opkg install /root/custom-packages/jansson_2.10-1_mips_24kc.ipk
opkg install /root/custom-packages/jose_10-1_mips_24kc.ipk
opkg install /root/custom-packages/libhttp-parser_2.8.0-1_mips_24kc.ipk
opkg install /root/custom-packages/tang_6-1_mips_24kc.ipk
\end{lstlisting}
The opkg tool will resolve other known dependencies and install them as well.
After packages are installed we may now proceed to the enviroment setup.



\section{Setting Up the Tang Keys}
The Tang server is packaged with cripts which help us to generate keys and cache for the Tang daemon.
OpenWrt has no realpath available, therefore the {\tt tangd-update} script did not work on OpenWrt.
\begin{lstlisting}[columns=fixed,basicstyle=\ttfamily\footnotesize,tabsize=4,backgroundcolor=\color{yellow!10}]
bash: realpath: command not found
\end{lstlisting}
Simply replacing it with {\tt readlink -f} solved the issue and script with such change is capable of generating cache.
We have to create a diff file for this change and add it into {\tt utils/tang/patches} directory.

The OpenWrt Makefile can define section {\it Package/\$(PKG\_NAME)/postinst}.
This section usually contain a short shell script to tweak the package after installation to make package work out of the box.
We can place this short sript to the Tang's Makefile to run the keys and cache generation after installation:
\begin{lstlisting}[columns=fixed,basicstyle=\ttfamily\footnotesize,tabsize=4,backgroundcolor=\color{yellow!10}]
define Package/tang/postinst
#!/bin/sh
if [ -z "$${IPKG_INSTROOT}" ]; then
	mkdir -p /usr/share/tang/db && mkdir -p /usr/share/tang/cache
	KEYS=$(find /usr/share/tang/db/ -name "*.jw*" -maxdepth 1 | wc -l)
	if [ "${KEYS}" = "0" ]; then # if db is empty generate new key pair
		/usr/libexec/tangd-keygen /usr/share/tang/db/
	elif [ "${KEYS}" = "1" ]; then # having 1 key should not happen
		(>&2 echo "Please check the Tang's keys in /usr/share/tang/db \
and regenate cache using /usr/libexec/tangd-update script.")
	else
		/usr/libexec/tangd-update /usr/share/tang/db/ /usr/share/tang/cache/
	fi
fi
endef
\end{lstlisting}



\section{Configure Tang for Xinetd}

We need xinetd's socket activation for Tang to work.
And to do so we will need a configuration file for the tangd service for xinetd daemons shown in listing \ref{tangdx}.
\begin{lstlisting}[columns=fixed,basicstyle=\ttfamily\footnotesize,tabsize=4,backgroundcolor=\color{yellow!10},caption=Configuration of Tang service for xinetd,label=tangdx]
service tangd
{
    port            = 8888
    socket_type     = stream
    wait            = no
    user            = root
    server          = /usr/libexec/tangd
    server_args     = /usr/share/tang/cache
    log_on_success  += USERID
    log_on_failure  += USERID
    disable         = no
}
\end{lstlisting}
This is configuration to run {\tt /usr/libexec/tangd} after a request comes to port 8888.
The server will be spawned with one argument, the cache directory.
Please note that we used diffent directory compared to Fedora.
We need to have Tang keys stored on persistent data storage.
The {\tt /var/} location is only a symbolic link to the {\tt /tmp} directory on OpenWrt device.
The developers on IRC channel proposed to use the persistent {\tt /usr/share/} directory for that purpose.
The last step before starting the Tang service on OpenWrt would be to setup {\tt /etc/services}:
\begin{lstlisting}[columns=fixed,basicstyle=\ttfamily\footnotesize,tabsize=4,backgroundcolor=\color{yellow!10}]
# echo -e "tangd\t\t8888/tcp" >> /etc/services
\end{lstlisting}
and add the very same thing to post-installation script to have Tang completelly set up after installation:
\begin{lstlisting}[columns=fixed,basicstyle=\ttfamily\footnotesize,tabsize=4,backgroundcolor=\color{yellow!10}]
(cat /etc/services | grep -E "tangd.*8888\/tcp") > /dev/null \
    || echo -e "tangd\t\t8888/tcp" >> /etc/services
\end{lstlisting}
In case of the accidental removal of {\tt /etc/services} file we have to copy the backup of the file from devices ROM and edit it again.
\begin{lstlisting}[columns=fixed,basicstyle=\ttfamily\footnotesize,tabsize=4,backgroundcolor=\color{yellow!10}]
# cp /rom/etc/services /etc/services
\end{lstlisting}
Now we shall restart the xinetd daemon using the xinetd script located in {\tt /etc/init.d/} directory to enable tangd service using xinetd's socket activation:
\begin{lstlisting}[columns=fixed,basicstyle=\ttfamily\footnotesize,tabsize=4,backgroundcolor=\color{yellow!10}]
# /etc/init.d/xinetd stop
# /etc/init.d/xinetd start
\end{lstlisting}
To test that service is running we can use the telnet to the device on port defined in the xinetd configuration.
The server should advertise its public key.
It can be retrieved with simple {\it GET} request for {\it /adv} content from the server.
Try telnet to device write “GET /adv HTTP/1.1” and confirm this with two newlines (return key).
The output similar to following should appear:
\begin{lstlisting}[columns=fixed,basicstyle=\ttfamily\footnotesize,tabsize=4,backgroundcolor=\color{yellow!10}]
$ telnet 192.168.0.1 8888
Trying 192.168.0.1...
Connected to 192.168.0.1.
Escape character is '^]'.
GET /adv HTTP/1.1

<unknown> GET /adv => 200 (src/tangd.c:85)
HTTP/1.1 200 OK
Content-Type: application/jose+json
Content-Length: 956

{"payload":"eyJrZXlzIjpbeyJhbGciOiJFQ01SIiwiY3J2IjoiUC01MjEiLCJrZXlfb3BzIjpb
ImRlcml2ZUtleSJdLCJrdHkiOiJFQyIsIngiOiJBR3V2amxUZmpYaDBraWFEa19Tak1vMGhYUm1R
dzFZNkVkNE9yN3Fza2J2c1h6QUlvRTl5ZnRwR2xRVng1OVlxZ1gtR3hQSE8tdzVLVXFmanRGQkVV
ZVByIiwieSI6IkFiNE9NNTBhQ1Y4NkdZVW1PdHBua1VTanNXNUFleENXZG5PZFEyakl0Z1RGNXNq
MG1TSFZCYXhsd2w1N3ZTUEdrSExiRl96SFlUVzlvVzNoTFJyeXRRNmYifSx7ImFsZyI6IkVTNTEy
IiwiY3J2IjoiUC01MjEiLCJrZXlfb3BzIjpbInZlcmlmeSJdLCJrdHkiOiJFQyIsIngiOiJBQVpS
ZmNlNUhLaFYyck1OQzhqLW5iVW1pdlh4NDRjcU1qX2Jvbk5OdnNTcXdBRjhFcGoyTFM0cFpfdUNR
VDJGVDRUSHJnX1Y4VHBKci1PQW41Z05CaUZNIiwieSI6IkFTMjJSdVJZZXVOU1ZtSkdfSmcwSW1n
by1LREd2QVFPZV9Vdk9fT3RZS3lGeUVoSWI1U0FLd3cwSEF0QjJFX2FTMG5pcWV3UUlod1QyanR5
eklCaWdkQU0ifV19","protected":"eyJhbGciOiJFUzUxMiIsImN0eSI6Imp3ay1zZXQranNvb
iJ9","signature":"ADpGOjYeB45MznQOxA6Pw9MXTMYQ649UkRSi_RsP8KPKosl-eA7GmOIiBM
FOoPnCNX-cGjhyDBbQuvESUCJ_3txjANf-srntxFAX5p72Eip-kROGrlCdLrVLnlg36itQUBFx7S
UyB_7I6CX6gdpWyJ-wtro8f2Snu6wwMGl-8V3ylWYr"}
\end{lstlisting}
The content of the first line from server response should be suspicious to us.
According to the RFC 2616 the HTTP header should not contain such line\cite{RFC2616}.
\begin{lstlisting}[columns=fixed,basicstyle=\ttfamily\footnotesize,tabsize=4,backgroundcolor=\color{yellow!10}]
<unknown> GET /adv => 200 (src/tangd.c:85)
\end{lstlisting}
After some investigation done in the Tang sources we found out that the line contains debug information from the Tang daemon which were written to the /dev/stderr.
From this we can assume that xinetd is sending also {\tt /dev/stderr} to the network socket.
To bypass this beavior we should write wrapper script which will redirect the {\tt /dev/stderr} output somewhere else.
We decided that this output may be helpful for debugging purpose and forwarded it to the log file in the {\tt /var/log/} directory of the device using this script:
\begin{lstlisting}[columns=fixed,basicstyle=\ttfamily\footnotesize,tabsize=4,backgroundcolor=\color{yellow!10}]
#!/bin/bash
echo "==================================" >> /var/log/tangd.log
echo `date`: >> /var/log/tangd.log
/usr/libexec/tangd $1 2>> /var/log/tangd.log
\end{lstlisting}
Let us name this wrapper script the tangdw.

To get Tang to work we need to put invocation of this script in place where tangd binary was configured to.
Move the script to the /usr/libexec directory where the tangd binary is to have it in one place and edit the line of xinetd configuration to run wrapper:
\begin{lstlisting}[columns=fixed,basicstyle=\ttfamily\footnotesize,tabsize=4,backgroundcolor=\color{yellow!10}]
    server          = /usr/libexec/tangdw
\end{lstlisting}
The last thing would be to change the script's permissions on device to allow execution.

We did a lot changes to the configuration.
The Tang for the OpenWrt platform needs a xinetd configuration which for the ease of use could be packaged with it.
The same with the wrapper script in order to have the service working correctly we also need this file.
There is a way to add new files for each package and package them for the OpenWrt's opkg to install.
To do so we will crate a directory for the files in the package's feeds repository we forked on branch containing commit adding the Tang and copy files there.
\begin{lstlisting}[columns=fixed,basicstyle=\ttfamily\footnotesize,tabsize=4,backgroundcolor=\color{yellow!10}]
$ cd packages-OpenWrt
$ git checkout add-tang
$ mkdir -p utils/tang/files
\end{lstlisting}
As the new files are copied there we need to also edit a {\it Package/tang/install} section to install these files in proper directories (config file into dedicated {\tt /etc/xinetd.d/}; wrapper into {\tt /usr/libexec/}) as shown:\newpage
\begin{lstlisting}[columns=fixed,basicstyle=\ttfamily\footnotesize,tabsize=4,backgroundcolor=\color{yellow!10}]
define Package/tang/install
	$(INSTALL_DIR)	$(1)/usr/libexec
	$(INSTALL_DIR)	$(1)/etc/xinetd.d/
	$(INSTALL_BIN)	\
			$(PKG_INSTALL_DIR)/usr/libexec/$(PKG_NAME)d*  $(1)/usr/libexec/
	$(INSTALL_BIN)	./files/tangdw	$(1)/usr/libexec/
	$(CP)			./files/tangdx	$(1)/etc/xinetd.d/
endef
\end{lstlisting}
After all changes done last, we need to to amend commit on a branch, update the feeds in buildroot, rebuild the Tang package, upload and re-install it on our device running the OpenWrt.
Due to many changes to the Makefile and the Tang branch in packages see the submitted pull-request\footnote{https://github.com/openwrt/packages/pull/5447} which includes all the changes.



\section{Binding to the OpenWrt Device Running Tang}
Let us now bind the client to the Tang server which is up and running on our OpenWrt:
\begin{lstlisting}[columns=fixed,basicstyle=\ttfamily\footnotesize,tabsize=4,backgroundcolor=\color{yellow!10}]
# clevis luks bind -d /dev/nvme0n1p2 tang '{"url":"http://192.168.0.1:8888"}'
The advertisement contains the following signing keys:

Apb39FO1vey9FyUe_fEd8lVDABs

Do you wish to trust these keys? [ynYN] y
Enter existing LUKS password:
\end{lstlisting}
Clevis will add a new key encryption key to the first available {\it LUKS} header key slot.
After rebuilding the initramfs on our Fedora client the early boot decryption will work.

Clevis client does not have option to unbind encrypted volume bound to Tang server.
We have to make sure that the key encryption key we are about to remove from the {\it LUKS} header key slot is one added by Clevis.
To do so please run {\tt cryptsetup} with option {\it luksDump} before and after binding with the Tang.
\begin{lstlisting}[columns=fixed,basicstyle=\ttfamily\footnotesize,tabsize=4,backgroundcolor=\color{yellow!10}]
# cryptsetup luksDump /dev/nvme0n1p2
\end{lstlisting}
The manual step to remove key from key slot was demonstrated in subsection \ref{manageLUKS}:
\begin{lstlisting}[columns=fixed,basicstyle=\ttfamily\footnotesize,tabsize=4,backgroundcolor=\color{yellow!10}]
# cryptsetup luksKillSlot /dev/nvme0n1p2 1
Enter any remaining passphrase:
\end{lstlisting}

\section{Tang's Limitations}\label{limitations}

Let us sum up the limitations that Tang has on OpenWrt platform due to replaced systemd dependency with less capable xinetd implementation of super-server.
We also found one platform independent limitation of the Tang's solution to full disk description on early boot.

\paragraph{Key exchange and generation} of cache must be manual but  can be automated with inotifytools to watch directory for change and run update script automatically.
But this mean that the embedded device with such configuration will have another process always running.
We decided to not implement automated update of the cache to save the computation resources.

\paragraph{xinetd forwarding the stderr} output to the socket as well is a little problem.
To have tang working correctly we wrote a wrapper running the shell script redirecting the {\it stderr} into log file.
Running some shell script takes some time compared to only running the binary file but for xinetd it is necessary.

\paragraph{Early boot decryption using Wi-Fi network} is a platform independent chicken-egg problem caused by information about wireless network being encrypted on system volume.
The information needed to connect to wireless network needs to be decrypted but in order to decrypt them automatically we need to connect to network exposed to Tang server.
It could be solved using a TPM to store information about such network but it is a security vulnerability and must be considered with a caution.
