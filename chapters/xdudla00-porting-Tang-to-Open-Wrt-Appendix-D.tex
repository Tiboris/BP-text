\chapter{Setting up the repository}\label{set_repo}

To save our work progress and be able to contribute to the upstream repository we should have a own fork of it.
A fork is a copy of a repository that we can manage.
It lets us make changes to a project without affecting the original (upstream) repository.
We can fetch updates from the upstream or submit changes to the original repository with pull requests.
These pull request are generated from the "devel" branch that we should have in our fork.
% Figure \ref{fig_it-workflow} GitHub's workflow reflects this in a nutshell.
% \begin{figure}[h]
%     \centering
%     \includegraphics[scale=1.3]{figures/placeholder.pdf}
%     \caption{GitHub's workflow}
%     \label{fig_it-workflow}
% \end{figure}

To fork OpenWrt's buildroot we should have our GitHub account set up\footnote{https://github.com/join}, visit the OpenWrt's project upstream repository\footnote{https://github.com/openwrt/openwrt}, and click on "Fork" button in the top-right corner of the page.
Before cloning our fork it is recomended to set ut git enviroment\footnote{https://git-scm.com/book/en/v2/Getting-Started-First-Time-Git-Setup} on host machine and upload the SSH keys\footnote{https://help.github.com/articles/adding-a-new-ssh-key-to-your-github-account/} to our account to minimize pushing effort to our fork.
All work related to this thesis has been done using GitHub account Tiboris\footnote{https://github.com/Tiboris/} therefore output of following commands are bound to it.
\begin{lstlisting}[columns=fixed,basicstyle=\ttfamily\footnotesize,basicstyle=\ttfamily\footnotesize,tabsize=4,backgroundcolor=\color{yellow!10}]
$ git clone https://github.com/Tiboris/openwrt.git ~/buildroot-openwrt
\end{lstlisting}
Please note that this command will clone the {\it openwrt} repository of the GitHub user {\it Tiboris} to the {\it $\sim$/buildroot-openwrt} directory on our host.

To catch up with changes made upstream we should consider to configure a remote that points to it.
Configured remote allows us to sync changes made in the original repository with the fork and vice versa.
Make sure you are in the buildroot directory and add remote\footnote{https://help.github.com/articles/configuring-a-remote-for-a-fork/} called upstream:
\begin{lstlisting}[columns=fixed,basicstyle=\ttfamily\footnotesize,basicstyle=\ttfamily\footnotesize,tabsize=4,backgroundcolor=\color{yellow!10}]
$ cd ~/buildroot-openwrt
$ git git remote add upstream \
     https://github.com/openwrt/openwrt.git
\end{lstlisting}
To verify that remote is present run:
\begin{lstlisting}[columns=fixed,basicstyle=\ttfamily\footnotesize,basicstyle=\ttfamily\footnotesize,tabsize=4,backgroundcolor=\color{yellow!10}]
$ git remote -v
origin	git@github.com:Tiboris/openwrt.git (fetch)
origin	git@github.com:Tiboris/openwrt.git (push)
upstream	https://github.com/openwrt/openwrt.git (fetch)
upstream	https://github.com/openwrt/openwrt.git (push)
\end{lstlisting}
With the remote conficured correctly we can now sync with
 upstream repository.
To download latest upstream changes use the "fetch" option
 and  the "merge" option to aply them to our fork's branch\footnote{https://help.github.com/articles/syncing-a-fork/}.
\begin{lstlisting}[columns=fixed,basicstyle=\ttfamily\footnotesize,basicstyle=\ttfamily\footnotesize,tabsize=4,backgroundcolor=\color{yellow!10}]
$ git fetch upstream master
remote: Counting objects: 4376, done.
remote: Compressing objects: 100% (2/2), done.
remote: Total 4376 (delta 1776), reused 1775 (delta 1775),
 pack-reused 2599
Receiving objects: 100% (4376/4376), 1.27 MiB
 | 743.00 KiB/s, done.
Resolving deltas: 100% (2932/2932), completed
 with 809 local objects.
From https://github.com/openwrt/openwrt
 * branch                  master     -> FETCH_HEAD
   36fb0697e2..d089a5d773  master     -> upstream/master
$ git merge upstream/master
\end{lstlisting}
The detailed description can be found on original GitHub pages\footnote{https://help.github.com/articles/working-with-forks/}.
