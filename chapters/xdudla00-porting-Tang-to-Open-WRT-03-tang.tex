\documentclass[../xdudla00-porting-Tang-to-Open-WRT.tex]{subfiles}
\begin{document}



\chapter{Tang}\label{tang}


Tang server is open source project implemented in C programming language, and it bind data to network presence.
It rely on few other software libraries:

\begin{itemize}
\item http-parser \ref{parser}
\item systemd / xinetd \ref{systemd}
\item jose \ref{jose}
    \begin{itemize}
    \item jansson \ref{jansson}
    \item openssl \ref{openssl}
    \item zlib \ref{zlib}
    \end{itemize}
\end{itemize}

Now when we know what Tang rely on and what is for we should be more specific about it and find out how Tang provides its service.
What binding data to network presence really means? Essentially it allows us to make some data to be available only when the system containing the data is on a certain, usually secure, network.

Tang server advertises asymmetric keys and client is able to get list off these signing keys \ref{jose} by HTTP GET request.
Now follows the provisioning step. With list of these public keys may process of encrypting data start. Client chooses one of asymmetric keys to generate a unique encryption key with.
After this client encrypt data using created key. Once the data is encrypted, the key is discarded.
Some small metadata has to be produced as part of this operation. The client should store these metadata in a convenient location.
Finally when client want to access encrypted data it must be able to recover encryption key.
This step starts with loading stored metadata and ends with simply performing a HTTP POST to Tang server.

Before Tang, automated decryption was handled usually with key escrow.
Client using key escrow usually generates a key, encrypt data with it and then stores the key in a remote server.
Key escrow server (also known as a “fair” cryptosystem) stores decryption keys. This should signal potential security risks when storing and also transmitting keys over network.
For security purposes key escrow uses SSL/TLS to provide privacy and data integrity between server and client.
Even with SSL/TLS key escrow requires authentication.
\todo{Treba to doplnit...}
To sum up an authorized third party may gain access to those keys under certain circumstances only.

\begin{table}[h]
\centering
\label{compare}
\begin{tabular}{@{}lll@{}}
\toprule
               & Escrow   & Tang                         \\ \midrule
Stateless      & No       & Yes                          \\
X.509          & Required & Optional                     \\
SSL/TLS        & Required & Optional                     \\
Authentication & Required & Optional                     \\
Anonymous      & No       & Yes                          \\ \bottomrule
\end{tabular}
\caption{Comparing Escrow and Tang}
\end{table}
\todo{popisat...}


Tang is originally packaged for Fedora OS version 24 and later but we can build it from source of course.\todo{doplnic...}

To build it from source download source from poject's GitHub or clone~it.

Make sure you have all needed dependencies installed and then run:

{\tt \$ autoreconf -if}

{\tt \$ ./configure --prefix=/usr}

{\tt \$ make}

{\tt \$ sudo make install}

Optionally to run tests:

{\tt \$ make check}

\section{http-parser}\label{parser}
Tang uses this is parser for both parsing HTTP requests and HTTP responses. 
You can find this parser on its own GitHub \cite{http-parser}.

\section{systemd / xinetd}\label{systemd}
systemd \cite{systemd} is a suite of basic building blocks for a Linux system. It provides a system and service manager that runs as PID 1 and starts the rest of the system. systemd provides aggressive parallelization capabilities, uses socket and D-Bus activation for starting services, offers on-demand starting of daemons, keeps track of processes using Linux control groups, maintains mount and automount points, and implements an elaborate transactional dependency-based service control logic.

\section{José}\label{jose}
José \cite{Jos} is a C-language implementation of the Javascript Object Signing and Encryption standards. Specifically, José aims towards implementing the following standards:
\begin{itemize}
   \item RFC 7515 - JSON Web Signature (JWS)        \cite{JWS}
   \item RFC 7516 - JSON Web Encryption (JWE)       \cite{JWE}
   \item RFC 7517 - JSON Web Key (JWK)              \cite{JWK}
   \item RFC 7518 - JSON Web Algorithms (JWA)       \cite{JWA}
   \item RFC 7519 - JSON Web Token (JWT)            \cite{JWT}
   \item RFC 7520 - Examples of ... JOSE            
   \item RFC 7638 - JSON Web Key (JWK) Thumbprint   \cite{JWK}
\end{itemize}

JOSE (Javascript Object Signing and Encryption) is a framework intended to provide a method to securely transfer claims (such as authorization information) between parties. 

Tang uses JWKs in comunication between client and server. Both POST request and reply bodies are JWK objects.

\subsection{jansson}\label{jansson}
Jansson \cite{jansson}(licenced under MIT licence)is a C library for encoding, decoding and manipulating JSON data. It features:
\begin{itemize}
   
    \item Simple and intuitive API and data model
    \item Comprehensive documentation
    \item No dependencies on other libraries
    \item Full Unicode support (UTF-8)
    \item Extensive test suite
\end{itemize}

\subsection{OpenSSL}\label{openssl}
OpenSSL \cite{openssl} contains an open-source implementation of the Transport Layer Security (TLS) and Secure Sockets Layer (SSL) protocols. It is used by network applications to secure communication between two parties over network.


\subsection{zlib}\label{zlib}
Library zlib \cite{zlib} is used for data compression.
\todo{zmazat ako podkapitoly?}

\end{document}