\documentclass[../xdudla00-porting-Tang-to-Open-WRT.tex]{subfiles}

\begin{document}

\chapter{Tang server}\label{tang} \todo{re-factor entire chapter}
Tang server is an open source project implemented in C programming language, and it binds data to network presence.
What does binding data to network presence really mean?
Essentially, it allows us to make some data to be available only when the system containing the data is on a particular, usually secure, network.

\section{Tang - binding daemon}

Tang server advertises asymmetric keys and a client is able to get the list of these signing keys \ref{jose} by HTTP (Hypertext Transfer Protocol) GET request.
The next step is the provisioning step. With the list of these public keys the process of encrypting data may start.
A client chooses one of the asymmetric keys to generate a unique encryption key.
After this, the client encrypts data using the created key. Once the data is encrypted, the key is discarded.
Some small metadata have to be produced as a part of this operation. The client should store these metadata to work with it when decrypting.

Finally, when the client wants to access the encrypted data, it must be able to recover encryption key.
This step starts with loading the stored metadata and ends with simply performing a HTTP POST to Tang server.
Server performs its mathematical operation and sends the result back to the client.
Finally, the client has to calculate the key value, which is better than when server calculates it.
So the Tang server never knew the value of the key and literraly nothing about its clients.

\begin{figure}[h]
    \centering
    \includegraphics[scale=0.7]{figures/TangModel.pdf}
    \caption{Tang model}
    \label{fig:tangmodel}
\end{figure}

On Figure \ref{fig:tangmodel} you can see the Tang model.
It is very similar to Escrow model \ref{fig:escrowmodel} but there are some thing missing.
In fact, there is no longer a need for TLS channel to secure comunication between the client and the server,
 and that is the reason why Tang implements the McCallum-Relyea exchange \ref{mrexchange} as described below.

\section{Binding with Tang}

A client performs an ECDH key exchange using the McCallum-Relyea algorithm \ref{mrexchange} in order to generate the binding key.
Then the client discards its own private key so that the Tang server is the only party that can reconstitute the binding key.
To blind \todo{blind?} the client's public key and the binding key, Tang uses a third, ephemeral key.
Ephemeral key is generated for each execution of a key establishment process.
Now only the client can unblind his public key and binding key.
\todo{ARROWS}
\begin{table}[h]
\centering
\label{mrexchange}
\begin{tabular}{c|c|c|c}
\hline
\multicolumn{2}{c|}{Provisioning} & \multicolumn{2}{c}{Recovery} \\ \hline  used
client's side & server's side & client's side & server's side \\ \hline
 & $ S \epsilon _{R} [1, p-1]$ & $E \epsilon _{R} [1, p-1]$ &  \\
 & $s = gS$ &$ x = c + gE$ &  \\
 & $\leftarrow$  s &  x $\rightarrow$ &  \\
$C \epsilon _{R} [1, p-1]$ &  &  & $y = zS $\\
$e = gC $&  &  & $\leftarrow$ y \\
$K = gSC = sC$ &  & $K = y - sE $&  \\
Discard: K, C &  &  &  \\
Retain s, c &  &  &  \\ \hline
\end{tabular}
\caption{McCallum-Relyea exchange}
\end{table}

\section{Provisioning}
The client selects one of the Tang server's exchange keys (we will call it sJWK; identified by the use of deriveKey in the sJWK's key\_ops attribute).
The lowercase "s" stands for server's key pair and JWK is used format of the message.
The client generates a new (random) JWK (cJWK; c stands for client's key pair).
The client performs its half of a standard ECDH exchange producing dJWK which it uses to encrypt the data.
Afterwards, it discards dJWK and the private key from cJWK.

The client then stores cJWK for later use in the recovery step.
Generally speaking, the client may also store other data, such as the URL of the Tang server or the trusted advertisement signing keys.

\begin{equation}
    s = g * S
\end{equation}

\begin{equation}
    c = g * C
\end{equation}

\begin{equation}
    K = s * C
\end{equation}

\section{Recovery}
To recover dJWK after discarding it, the client generates a third ephemeral key (eJWK).
Using eJWK, the client performs elliptic curve group addition of eJWK and cJWK, producing xJWK. The client POSTs xJWK to the server.

The server then performs its half of the ECDH key exchange using xJWK and sJWK, producing yJWK. The server returns yJWK to the client.

The client then performs half of an ECDH key exchange between eJWK and sJWK, producing zJWK. Subtracing zJWK from yJWK produces dJWK again.

Mathematically (capital is private key; g stands for generate) client's operation:

\begin{equation}
    e = g * E
\end{equation}

\begin{equation}
    x = c + e
\end{equation}

\begin{equation}
    y = x * S
\end{equation}

\begin{equation}
    z = s * E
\end{equation}

\begin{equation}
    K = y - z
\end{equation}

\section{Security}

We can now compare Tang and Escrow. In contrast, Tang is stateless and doesn't require TLS or authentication.
Tang also has limited knowledge. Unlike escrows, where the server has knowledge of every key ever used, Tang never sees a single client key.
Tang never gains any identifying information from the client.

\begin{table}[h]
\centering
\label{compare}
\begin{tabular}{@{}lll@{}}
\toprule
               & Escrow   & Tang                         \\ \midrule
Stateless      & No       & Yes                          \\
SSL/TLS        & Required & Optional                     \\
X.509          & Required & Optional                     \\
Authentication & Required & Optional                     \\
Anonymous      & No       & Yes                          \\ \bottomrule
\end{tabular}
\caption{Comparing Escrow and Tang}
\end{table}

Let's think about the security of Tang system. Is it really secure without an encrypted channel or even without authentication?
So long as the client discards its private key, the client cannot recover dJWK without the Tang server.
This is fundamentally the same assumption used by Diffie-Hellman (and ECDH).

\subsection{Man-in-the-Middle attack}
In this case, the eavesdropper in this case sees the client send xJWK and receive yJWK.
Since, these packets are blinded by eJWK, only the party that can unblind these values is the client itself (since only it has eJWK's private key).
Thus, the MitM attack fails.
\subsection{Compromise the client to gain access to cJWK}
It is of utmost importance that the client protects cJWK from prying eyes.
This may include device permissions, filesystem permissions, security frameworks (such as SELinux - Security-Enhanced Linux) or even the use of hardware encryption such as a TPM.
How precisely this is accomplished depends on the client implementation.
\subsection{Compromise the server to gain access to sJWK's private key}
The Tang server must protect the private key for sJWK.
In this implementation, access is controlled by file system permissions and the service's policy.
An alternative implementation might use hardware cryptography (for example, an HSM) to protect the private key.
\section{Building Tang}

Tang is originally packaged for Fedora OS version 23 and later but we can build it from source of course.
It relies on few other software libraries:
\label{dependencies}
\begin{itemize}
\item http-parser \ref{http-parser}
\item systemd / xinetd \ref{systemd}
\item jose \ref{jose}
    \begin{itemize}
    \item jansson \ref{jansson}
    \item openssl \ref{openssl}
    \item zlib \ref{zlib}
    \end{itemize}
\end{itemize}

The steps to build it from source include download source from poject's GitHub or clone~it.
Make sure you have all needed dependencies installed and then run:

{\tt \$ autoreconf -if}

{\tt \$ ./configure --prefix=/usr}

{\tt \$ make}

{\tt \$ sudo make install}

Optionally to run tests:

{\tt \$ make check}

\subsection{http-parser}\label{http-parser}
Tang uses this parser for both parsing HTTP requests and HTTP responses.
The parser can be found on its own GitHub \cite{http-parser}.

\subsection{systemd / xinetd}\label{systemd}
systemd is a suite of basic building blocks for a Linux system.
It provides a system and service manager that runs as PID 1 and starts the rest of the system.
systemd provides aggressive parallelization capabilities, uses socket and D-Bus activation for starting services,
 offers on-demand starting of daemons, keeps track of processes using Linux control groups, maintains mount and automount points,
 and implements an elaborate transactional dependency-based service control logic.
\todo{Why is systemd needed by tang}

\subsection{José}\label{jose}
José \cite{jose_prog} is a C-language implementation of the Javascript Object Signing and Encryption standards.
Specifically, José aims towards implementing the following standards:
\begin{itemize}
   \item RFC 7515 - JSON Web Signature (JWS)        \cite{JWS}
   \item RFC 7516 - JSON Web Encryption (JWE)       \cite{JWE}
   \item RFC 7517 - JSON Web Key (JWK)              \cite{JWK}
   \item RFC 7518 - JSON Web Algorithms (JWA)       \cite{JWA}
   \item RFC 7519 - JSON Web Token (JWT)            \cite{JWT}
   \item RFC 7520 - Examples of ... JOSE            %\cite{RFC7520}
   \item RFC 7638 - JSON Web Key (JWK) Thumbprint   \cite{JWK}
\end{itemize}

JOSE (Javascript Object Signing and Encryption) is a framework intended to provide a method to securely transfer claims (such as authorization information) between parties.

Tang uses JWKs in comunication between client and server. Both POST request and reply bodies are JWK objects.

\subsection{jansson}\label{jansson}
Jansson \cite{jansson}(licenced under MIT licence) is a C library for encoding, decoding and manipulating JSON data. It features:
\begin{itemize}

    \item Simple and intuitive API and data model
    \item Comprehensive documentation
    \item No dependencies on other libraries
    \item Full Unicode support (UTF-8)
    \item Extensive test suite
\end{itemize}

\subsection{OpenSSL}\label{openssl}
OpenSSL contains an open-source implementation of the Transport Layer Security (TLS) and Secure Sockets Layer (SSL) protocols.
It is used by network applications to secure communication between two parties over network.

\subsection{zlib}\label{zlib}
Library zlib \cite{zlib} is used for data compression.

\section{Server enablement}
Enabling a Tang server is a two-step process.
First, enable and start the service using systemd.

{\tt\begin{verbatim} $ sudo systemctl enable tangd-update.path\end{verbatim}
}

{\tt\begin{verbatim} $ sudo systemctl start tangd-update.path\end{verbatim}
}

{\tt\begin{verbatim} $ sudo systemctl enable tangd.socket\end{verbatim}
}

{\tt\begin{verbatim} $ sudo systemctl start tangd.socket\end{verbatim}
}

Second, generate a signing key and an exchange key.

{\tt\begin{verbatim} $ sudo jose gen -t '{"alg":"ES256"}' -o /var/db/tang/sig.jwk\end{verbatim}
}

{\tt\begin{verbatim} $ sudo jose gen -t '{"kty":"EC","crv":"P-256","key_ops":["deriveKey"]}' \
        -o /var/db/tang/exc.jwk\end{verbatim}
}

Now we are up and running. Server is ready to send advertisment on demand.
\todo{Get clevis in here?}


\section{Clevis client}\label{clevis}

Clevis provides a pluggable key management framework for automated decryption.
It can handle even automated unlocking of LUKS volumes.
To do so, we have to encrypt some data with simple command:

{\tt\begin{verbatim} $ clevis encrypt PIN CONFIG < PLAINTEXT > CIPHERTEXT.jwe\end{verbatim}
}

In clevis terminology, a {\it pin} is a plugin which implements automated decryption.
We simply pass the name of supported pin here.
Secondly {\it config} is a JSON object which will be passed directly to the {\it pin}.
It contains all the necessary configuration to perform encryption and setup automated decryption.

\subsection{PIN: Tang}
Clevis has full support for Tang. Here is an example of how to use Clevis with Tang:
{\tt \begin{verbatim} $ echo hi | clevis encrypt tang '{"url": "http://tangserver"}' > hi.jwe
 The advertisement is signed with the following keys:
     kWwirxc5PhkFIH0yE28nc-EvjDY

 Do you wish to trust the advertisement? [yN] y\end{verbatim}
}
In this example, we encrypt the message "hi" using the Tang pin.
The only parameter needed in this case is the URL of the Tang server.
During the encryption process, the Tang pin requests the key advertisement from the server and asks you to trust the keys.
This works similarly to SSH.

Alternatively, you can manually load the advertisment using the adv parameter.
This parameter takes either a string referencing the file where the advertisement is stored, or the JSON contents of the advertisment itself.
When the advertisment is specified manually like this, Clevis presumes that the advertisement is trusted.
\subsection{PIN: HTTP}
Clevis also ships a pin for performing escrow using HTTP.
Please note that, at this time, this pin does not provide HTTPS support and is suitable only for use over local sockets.
This provides integration with services like Custodia.

\subsection{PIN: SSS - Shamir Secret Sharing}
Clevis provides a way to mix pins together to provide sophisticated unlocking policies.
This is accomplished by using an algorithm called Shamir Secret Sharing (SSS).

\subsection{Binding LUKS volumes}
Clevis can be used to bind a LUKS volume using a pin so that it can be automatically unlocked.

How this works is rather simple. We generate a new, cryptographically strong key. This key is added to LUKS as an additional passphrase. We then encrypt this key using Clevis, and store the output JWE inside the LUKS header using LUKSMeta.

Here is an example where we bind {\tt /dev/vda2} using the Tang ping:
{\tt \begin{verbatim} $ sudo clevis bind-luks /dev/sda1 tang '{"url": "http://tang.local"}'
 The advertisement is signed with the following keys:
         kWwirxc5PhkFIH0yE28nc-EvjDY

 Do you wish to trust the advertisement? [yN] y
 Enter existing LUKS password:\end{verbatim}
}
Upon successful completion of this binding process, the disk can be unlocked using one of the provided unlockers.

\subsection{Dracut}\label{dracut}
The Dracut unlocker attempts to automatically unlock volumes during early boot.
This permits automated root volume encryption.
Just rebuild your initramfs after installing Clevis:

{\tt \begin{verbatim} $ sudo dracut -f\end{verbatim}
}

Upon reboot, you will be prompted to unlock the volume using a password. In the background, Clevis will attempt to unlock the volume automatically. If it succeeds, the password prompt will be cancelled and boot will continue.

\subsection{UDisks2}\label{udisk2}
Our UDisks2 unlocker runs in your desktop session.
You should not need to manually enable it; just install the Clevis UDisks2 unlocker and restart your desktop session.
The unlocker should be started automatically.

This unlocker works almost exactly the same as the Dracut unlocker.
If you insert a removable storage device that has been bound with Clevis, we will attempt to unlock it automatically in parallel with a desktop password prompt.
If automatic unlocking succeeds, the password prompt will be dissmissed without user intervention.


\end{document}
