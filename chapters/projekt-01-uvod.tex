\documentclass[../projekt.tex]{subfiles}
\begin{document}
%=========================================================================
% (c) Michal Bidlo, Bohuslav Křena, 2008

\chapter{Úvod}\label{uvod}

Abychom mohli napsat odborný text jasně a~srozumitelně, musíme splnit několik základních předpokladů:
\begin{itemize}
\item Musíme mít co říci,
\item musíme vědět, komu to chceme říci,
\item musíme si dokonale promyslet obsah,
\item musíme psát strukturovaně. 
\end{itemize}

Tyto a další pokyny jsou dostupné též na školních internetových stránkách \cite{fitWeb}.

Přehled základů typografie a tvorby dokumentů s využitím systému \LaTeX je 
uveden v~\cite{Rybicka}.

Reference na Kapitolu Uvod ... \ref{uvod}.

\section{Musíme mít co říci}
Dalším důležitým předpokladem dobrého psaní je {\it psát pro někoho}. Píšeme-li si poznámky sami pro sebe, píšeme je jinak než výzkumnou zprávu, článek, diplomovou práci, knihu nebo dopis. Podle předpokládaného čtenáře se rozhodneme pro způsob psaní, rozsah informace a~míru detailů.

\section{Musíme vědět, komu to chceme říci}
Dalším důležitým předpokladem dobrého psaní je psát pro někoho. Píšeme-li si poznámky sami pro sebe, píšeme je jinak než výzkumnou zprávu, článek, diplomovou práci, knihu nebo dopis. Podle předpokládaného čtenáře se rozhodneme pro způsob psaní, rozsah informace a~míru detailů.

\section{Musíme si dokonale promyslet obsah}
Musíme si dokonale promyslet a~sestavit obsah sdělení a~vytvořit pořadí, v~jakém chceme čtenáři své myšlenky prezentovat. 
Jakmile víme, co chceme říci a~komu, musíme si rozvrhnout látku. Ideální je takové rozvržení, které tvoří logicky přesný a~psychologicky stravitelný celek, ve kterém je pro všechno místo a~jehož jednotlivé části do sebe přesně zapadají. Jsou jasné všechny souvislosti a~je zřejmé, co kam patří.

Abychom tohoto cíle dosáhli, musíme pečlivě organizovat látku. Rozhodneme, co budou hlavní kapitoly, co podkapitoly a~jaké jsou mezi nimi vztahy. Diagramem takové organizace je graf, který je velmi podobný stromu, ale ne řetězci. Při organizaci látky je stejně důležitá otázka, co do osnovy zahrnout, jako otázka, co z~ní vypustit. Příliš mnoho podrobností může čtenáře právě tak odradit jako žádné detaily.

Výsledkem této etapy je osnova textu, kterou tvoří sled hlavních myšlenek a~mezi ně zařazené detaily.

\section{Musíme psát strukturovaně} 
Musíme začít psát strukturovaně a~současně pracujeme na co nejsrozumitelnější formě, včetně dobrého slohu a~dokonalého značení. 
Máme-li tedy myšlenku, představu o~budoucím čtenáři, cíl a~osnovu textu, můžeme začít psát. Při psaní prvního konceptu se snažíme zaznamenat všechny své myšlenky a~názory vztahující se k~jednotlivým kapitolám a~podkapitolám. Každou myšlenku musíme vysvětlit, popsat a~prokázat. Hlavní myšlenku má vždy vyjadřovat hlavní věta a~nikoliv věta vedlejší.

I k~procesu psaní textu přistupujeme strukturovaně. Současně s~tím, jak si ujasňujeme strukturu písemné práce, vytváříme kostru textu, kterou postupně doplňujeme. Využíváme ty prostředky DTP programu, které podporují strukturovanou stavbu textu (předdefinované typy pro nadpisy a~bloky textu). 

\end{document}