\documentclass[../xdudla00-porting-Tang-to-Open-WRT.tex]{subfiles}
\begin{document}
\chapter{Clevis}\label{clevis}

Clevis provides a pluggable key management framework for automated decryption \cite{clevis}.
It can handle even automated unlocking of LUKS volumes.
To do so we have to encrypt some data with simple command:

{\tt\begin{verbatim} $ clevis encrypt PIN CONFIG < PLAINTEXT > CIPHERTEXT.jwe\end{verbatim}
}
In clevis terminology, a {\it PIN} is a plugin which implements automated decryption.
We simply pass the name of supported pin here.
Secondly {\it CONFIG} is a JSON object which will be passed directly to the {\it PIN}.
It contains all the necessary configuration to perform encryption and setup automated decryption.

\section{PIN: Tang}
Clevis has full support for Tang. Here is an example of how to use Clevis with Tang:
{\tt \begin{verbatim} $ echo hi | clevis encrypt tang '{"url": "http://tangserver"}' > hi.jwe
 The advertisement is signed with the following keys:
     kWwirxc5PhkFIH0yE28nc-EvjDY

 Do you wish to trust the advertisement? [yN] y\end{verbatim}
}
In this example, we encrypt the message "hi" using the Tang pin.
The only parameter needed in this case is the URL of the Tang server.
During the encryption process, the Tang pin requests the key advertisement from the server and asks you to trust the keys.
This works similarly to SSH.

Alternatively, you can manually load the advertisment using the adv parameter.
This parameter takes either a string referencing the file where the advertisement is stored, or the JSON contents of the advertisment itself.
When the advertisment is specified manually like this, Clevis presumes that the advertisement is trusted.
\section{PIN: HTTP}
Clevis also ships a pin for performing escrow using HTTP.
Please note that, at this time, this pin does not provide HTTPS support and is suitable only for use over local sockets.
This provides integration with services like Custodia.

\section{PIN: SSS - Shamir Secret Sharing}
Clevis provides a way to mix pins together to provide sophisticated unlocking policies.
This is accomplished by using an algorithm called Shamir Secret Sharing (SSS).

\section{Binding LUKS volumes}
Clevis can be used to bind a LUKS volume using a pin so that it can be automatically unlocked.

How this works is rather simple. We generate a new, cryptographically strong key. This key is added to LUKS as an additional passphrase. We then encrypt this key using Clevis, and store the output JWE inside the LUKS header using LUKSMeta.

Here is an example where we bind {\tt /dev/vda2} using the Tang ping:
{\tt \begin{verbatim} $ sudo clevis bind-luks /dev/sda1 tang '{"url": "http://tang.local"}'
 The advertisement is signed with the following keys:
         kWwirxc5PhkFIH0yE28nc-EvjDY

 Do you wish to trust the advertisement? [yN] y
 Enter existing LUKS password:\end{verbatim}
}
Upon successful completion of this binding process, the disk can be unlocked using one of the provided unlockers.

\subsection{Dracut}
The Dracut unlocker attempts to automatically unlock volumes during early boot. This permits automated root volume encryption. Enabling the Dracut unlocker is easy. Just rebuild your initramfs after installing Clevis:

{\tt \begin{verbatim} $ sudo dracut -f\end{verbatim}
}

Upon reboot, you will be prompted to unlock the volume using a password. In the background, Clevis will attempt to unlock the volume automatically. If it succeeds, the password prompt will be cancelled and boot will continue.

\subsection{UDisks2}
Our UDisks2 unlocker runs in your desktop session.
You should not need to manually enable it; just install the Clevis UDisks2 unlocker and restart your desktop session.
The unlocker should be started automatically.

This unlocker works almost exactly the same as the Dracut unlocker.
If you insert a removable storage device that has been bound with Clevis, we will attempt to unlock it automatically in parallel with a desktop password prompt.
If automatic unlocking succeeds, the password prompt will be dissmissed without user intervention.

\end{document}
