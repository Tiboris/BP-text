\chapter{Conclusion}\label{conlusion}



The Tang \ref{tang} server is a very lightweight program.
It provides secure and anonymous data binding using McCallum-Relyea exchange algorithm.

As every server purpose is to serve its clients, it needs to have client application.
In case of Tang we have Clevis.
Clevis is a client software with full support for Tang.
It has minimal dependencies and it is possible to use with HTTP, Escrow, and it implements Shamir Secret Sharing.
Clevis has GNOME integration so it is not only a command line tool.
Clevis also supports removable devices unlocking using UDisks2 or even early boot integration with dracut.
This was inspiration for this thesis with a goal to achieve it with OpenWrt embedded device running Tang server only.

To port Tang to OpenWrt system it was necessary to port all its dependencies first.

After struggling with older version, package http-parser known as libhttp-parser in OpenWrt feeds, is now updated to latest upstream version 2.8.0 with {\it STATUS\_MAP} patch.
The systemd would be huge effort but tang's requirements are minimal and we were able to work with xinetd's socket activation.
With correct configuration of xinetd and removing dependency for systemd Tang server is running on OpenWrt with some platform specific changes and limitations.
