\chapter{Atomated decryption}

People usually try to automate everything that could be and not bother with repetive tasks.
Even that disc enctyption provides another layer of security to our data it is not used often.
Typing one more passphrase when accessing some removable storage seems to bother.
More crucial would be full disk encryption.
In order to boot the system we need to have storage with system decrypted - provide another passphrase every time the computer is turned off and on again.
Tang server aims to solve struggles with the early boot decryption of system volumes.
Before Tang, automated decryption was usually handled by a Key Escrow server.

\section{Key Escrow}\label{escrow}

The Key Escrow server (also known as a “fair” cryptosystem) is providing escrow service for encryption keys.
A client using Key Escrow usually generates a key, encrypts data with it and then stores the key encryption key on a remote server.
Unfortunatelly, it is not always as simple as it sounds and there are couple of security concerns.

To transfer the encryption keys we want to store on an Escrow server, we have to encrypt the channel on which we send them.
When transmitting keys over an insecure network without an encrypted link, anyone listening to the network traffic could immadiately fetch the key.
This should signal security risks, and, of course, we do not want any third party to access our secret data.
Usually we encrypt a channel with TLS(Transport Layer Security) or GSSAPI (Generic Security Services Application Program Interface) as shown on a Figure \ref{fig_escrowmodel} Escrow model.
\begin{figure}[h]
    \centering
    \includegraphics[scale=0.7]{figures/EscrowModel.pdf}
    \caption{Escrow model}
    \label{fig_escrowmodel}
\end{figure}
Unfortunatelly, this is not enough to have the communication secure.
We can not just start sending these keys to the escrow server if we do not know whether this server is the one it acts to be.
This server has to have its identity verified, and the client has to authenticate to this server too.
The increasing amount of keys implicates a need for Certification Authority server (CA) or a Key Distribution Center (KDC) to manage all of them.

With all these keys, and at this point only, server can verify if the client is permitted to get their key, and the client is able to identify trusted server.
This is a fully stateful process.
To sum up, an authorized third party may gain access to keys stored on an Escrow server under certain circumstances only.

The complexity of this system increases the attack surface and for a such complex system it would be unimaginable not to have backups.
Escrow server may store lots of keys from lots of different places, users and services.

With key escrow a third party gets copies of a cryptographic key.
People might not be comfortable with any third party having this ability and that "technical" problems vex the key escrow solution.
Tang's key recovery, on the other hand, lets us just "backup" and restore cryptographic keys anonymously and without any third party possesing our key.

Key recovery is necessary.
Key escrow is not.
Let us not confuse these two.



\section{Tang Server}\label{tang}

Tang is a very lightweight network service using systemd's socket activation as described in section \ref{socket_activation} Socket activation.
Its purpose is to provide anonymous key recovery to clients all over the accessible network.
This key recovery can be used with a help of the Tang's client descibed in subsection \ref{clevis} Clevis client to unlock data storages with LUKS encryption.
Tang server is an open source project implemented in C programming language.
Let us see how it works.

We can see on Figure \ref{fig_tangmodel} Tang Model that model is very similar to the escrow seen on Figure \ref{fig_escrowmodel} Escrow model, but with some things missing.
\begin{figure}[h]
    \centering
    \includegraphics[scale=0.7]{figures/TangModel.pdf}
    \caption{Tang model}
    \label{fig_tangmodel}
\end{figure}
In fact, there is no longer a need for TLS channel to secure comunication between the client and the server,
 and that is because Tang implements the McCallum-Relyea exchange as described below.

Tang server advertises asymmetric keys on the network and a client is able to get the list of these signing keys by HTTP (Hypertext Transfer Protocol) GET request or even using curl:
\begin{lstlisting}[columns=fixed,basicstyle=\ttfamily\footnotesize,tabsize=4,backgroundcolor=\color{yellow!10}]
curl -f http://tang.local/adv
\end{lstlisting}
All tang comunication is performed using HTTP protocol with the message body in JWK(JSON Web Key) format defined by RFC 7517 \cite{RFC7517}.
Getting advertised keys is the first step of the McCallum-Relyea exchange protocol summed up in table \ref{mrexchange} McCallum-Relyea exchange protocol.
\begin{table}[h]
\centering
\label{mrexchange}
\begin{tabular}{c|c|c|c}
\hline
\multicolumn{2}{c|}{Provisioning} & \multicolumn{2}{c}{Recovery} \\ \hline
client's side & server's side & client's side & server's side \\ \hline
 & $ S \epsilon _{bindingR} [1, p-1]$ & $E \epsilon _{R} [1, p-1]$ &  \\
 & $s = g * S$ &$ x = c + g * E$ &  \\
\multicolumn{2}{c|} {$\leftarrow$  s} & \multicolumn{2}{c} {x $\rightarrow$}  \\
$C \epsilon _{R} [1, p-1]$ &  &  & $y = z * S$\\
$e = g * C $&  &  & $\leftarrow$ y \\
$K = g * S * C = s * C$ &  & $K = y - s * E$&  \\
Discard: K, C &  &  &  \\
Retain s, c &  &  &  \\ \hline
\multicolumn{4}{c}{*capital is private key; g stands for generate}
\end{tabular}
\caption{McCallum-Relyea exchange protocol}
\end{table}
This protocol has two phases provisioning and recovery.

\paragraph{Provisioning}
The server key pair generation is represented by equation \label{servergen}, capital is private key; lowercase is public key; g stands for generate.
\begin{equation}\label{servergen}
    s = g * S
\end{equation}
After the server generates key pair it is advertising the public part of it.
The client then selects one of the Tang server's exchange keys (we will call it sJWK; identified by the use of deriveKey in the sJWK's key\_ops attribute).
The lowercase "s" stands for server's public key and JWK is used format of the message.
The client then generates a new (random) JWK (cJWK; c stands for client's key pair).
\begin{equation}\label{clientgen}
    c = g * C
\end{equation}
The client performs its half of a standard ECDH exchange producing dJWK which it uses to encrypt the data.
Afterwards, it must discard the dJWK and the private key from cJWK.
The dJWK is represented as K in equation \ref{dJWK_gen}.
\begin{equation}\label{dJWK_gen}
    K = s * C
\end{equation}
The client have to store cJWK for later use in the recovery step.
Generally speaking, the client may also store other data, such as the URL of the Tang server or the trusted advertisement signing keys called also binding keys.



\paragraph{Recovery}
Mathematically capital is private key; g stands for generate.
When the client wants to access the encrypted data, it must be able to recover encryption key.
To recover dJWK after discarding it, the client generates a third ephemeral key (eJWK) as equation \ref{eqn_ephemeral} represent.
This key is used to hide the client's public key and the binding key.
\begin{equation}\label{eqn_ephemeral}
    e = g * E
\end{equation}
Ephemeral key is generated for each execution of a key establishment process particularly.
Using eJWK, the client performs elliptic curve group addition of eJWK and cJWK, producing xJWK represented in equation \ref{eqn_ecga}.
The client POSTs xJWK to the server.
\begin{equation}\label{eqn_ecga}
    x = c + e
\end{equation}
The server then performs its half of the ECDH key exchange using xJWK and sJWK, producing yJWK reflected in equation \ref{eqn_response}. The server returns yJWK to the client.
\begin{equation}\label{eqn_response}
    y = x * S
\end{equation}
The client then performs half of an ECDH key exchange between eJWK and sJWK, producing zJWK as equation \ref{eqn_get} shows.
\begin{equation}\label{eqn_get}
    z = s * E
\end{equation}
Subtracing zJWK from yJWK produces dJWK represented as K in equation \ref{eqn_extract} again.
\begin{equation}\label{eqn_extract}
    K = y - z
\end{equation}
Finally, the client has calculated the key value, which is better than when server calculates it.
So the Tang server never knew the value of the key and literraly nothing about its clients.



\subsection{Security}

In contrast shown in table \ref{compare} Comparing Escrow and Tang, Tang compared to Escrow is stateless and doesn't require TLS or authentication.
Tang also has limited knowledge.
Unlike escrows, where the server has knowledge of every key ever used, Tang never sees a single client key.
Tang never gains any identifying information from the client.

\begin{table}[h]
\centering
\label{compare}
\begin{tabular}{@{}lll@{}}
\toprule
               & Escrow   & Tang                         \\ \midrule
Stateless      & No       & Yes                          \\
SSL/TLS        & Required & Optional                     \\
X.509          & Required & Optional                     \\
Authentication & Required & Optional                     \\
Anonymous      & No       & Yes                          \\ \bottomrule
\end{tabular}
\caption{Comparing Escrow and Tang}
\end{table}

Thanks to McCallum-Relyea exchange protocol summed up in table \ref{mrexchange} McCallum-Relyea exchange protocol Tang is resistant to the man in the middle attack.
In case of the eavesdroppers they sees the client send xJWK and receive yJWK.
Since, these packets are blinded by eJWK, only the party that can unblind these values is the client itself (since only it has eJWK's private key).
Thus, the MitM attack fails.

It is of utmost importance that the client protects cJWK from prying eyes and the Tang server must protect the private key for sJWK.



\subsection{Building Tang}

Tang is originally packaged for Fedora OS version 23 and later but we can build it from source of course.
It relies on few other software libraries:
\label{dependencies}
\begin{itemize}
\item http-parser
\item systemd's socket activation
\item jose
    \begin{itemize}
    \item jansson
    \item openssl
    \item zlib
    \end{itemize}
\end{itemize}

The steps to build the Tang from sources include download source from poject's GitHub or clone~it.
Make sure we have all needed dependencies installed and then run:
\begin{lstlisting}[columns=fixed,basicstyle=\ttfamily\footnotesize,tabsize=4,backgroundcolor=\color{yellow!10}]
$ autoreconf -if
$ ./configure --prefix=/usr
$ make
# make install
\end{lstlisting}
Optionally to run tests:
\begin{lstlisting}[columns=fixed,basicstyle=\ttfamily\footnotesize,tabsize=4,backgroundcolor=\color{yellow!10}]
$ make check
\end{lstlisting}



\subsection{Server enablement}

Enabling a Tang server is a two-step process.
First step is to enable the Tang services.
Start the key update service which is watching database directory using systemd.
After database of the keys is changed the update service will regenerate the cache used by tang:
\begin{lstlisting}[columns=fixed,basicstyle=\ttfamily\footnotesize,tabsize=4,backgroundcolor=\color{yellow!10}]
# systemctl enable tangd-update.path --now
\end{lstlisting}
Enable service using systemd socket activation:
\begin{lstlisting}[columns=fixed,basicstyle=\ttfamily\footnotesize,tabsize=4,backgroundcolor=\color{yellow!10}]
# systemctl enable tangd.socket --now
\end{lstlisting}
After this systemd will handle the sockets and the Tang's key rotation procedure.

Secondly generate a signing key using dependency tool jose and store it in default directory expected by tang:
\begin{lstlisting}[columns=fixed,basicstyle=\ttfamily\footnotesize,tabsize=4,backgroundcolor=\color{yellow!10}]
# jose gen -t '{"alg":"ES256"}' -o /var/db/tang/sig.jwk
\end{lstlisting}
do not forget to generate an exchange key:
\begin{lstlisting}[columns=fixed,basicstyle=\ttfamily\footnotesize,tabsize=4,backgroundcolor=\color{yellow!10}]
# jose gen -t '{"kty":"EC","crv":"P-256","key_ops":["deriveKey"]}' \
        -o /var/db/tang/exc.jwk
\end{lstlisting}

Now we are up and running.
Keys are stored in Tang's database directory and update script activated with systemd generated the cache in .
Server is ready to send advertisment on demand.



\subsection{Clevis client}\label{clevis}

Clevis has full support for Tang and provides a pluggable key management framework for automated decryption.
It can handle automated unlocking of LUKS volumes.
Clevis let us try to encrypt some data with simple command:
\begin{lstlisting}[columns=fixed,basicstyle=\ttfamily\footnotesize,tabsize=4,backgroundcolor=\color{yellow!10}]
$ clevis encrypt PIN CONFIG < PLAINTEXT > CIPHERTEXT.jwe
\end{lstlisting}
In clevis terminology, a {\it PIN} is a plugin which implements automated decryption.
We simply pass the name of supported pin here.
Besides Tang {\it PIN} clevis also support a {\it PIN} for performing escrow using HTTP or

Secondly {\it CONFIG} is a JSON object which will be passed directly to the {\it PIN}.
It contains all the necessary configuration to perform encryption and setup automated decryption.

It also support a {\it PIN} for performing escrow using HTTP or a SSS {\it PIN} to provide a way to mix pins together to provide sophisticated unlocking policies by using an algorithm called Shamir Secret Sharing (SSS).


\paragraph{PIN: Tang}
Clevis has full support for Tang.
Here is an example of how to use Clevis with Tang:
\begin{lstlisting}[columns=fixed,basicstyle=\ttfamily\footnotesize,tabsize=4,backgroundcolor=\color{yellow!10}]
$ echo hi | clevis encrypt tang '{"url": "http://tang.local"}' > hi.jwe
The advertisement contains the following signing keys:

Apb39FO1vey9FyUe_fEd8lVDABs

Do you wish to trust these keys? [ynYN] y
$ clevis decrypt tang '{"url": "http://tang.local"}' < hi.jwe
hi
\end{lstlisting}
In this example, we encrypt the message "hi" using the Tang pin.
The only parameter needed in this case is the URL of the Tang server.
During the encryption process, the Tang pin requests the key advertisement from the server and asks us to trust the keys.
This works similarly to SSH.

Alternatively, we can manually load the advertisment using the adv parameter.
This parameter takes either a string referencing the file where the advertisement is stored, or the JSON contents of the advertisment itself.
When the advertisment is specified manually like this, Clevis presumes that the advertisement is trusted.



\subsection{Binding LUKS volumes with clevis}\label{dracut}
Tang's main goal is to solve early boot decryption of the LUKS volumes and clevis has solution for it.
Clevis can be used to bind a LUKS volume using a pin so that it can be automatically unlocked.

How this works is rather simple. We generate a new, cryptographically strong key.
This key is added to LUKS as an additional passphrase.
We then encrypt this key using Clevis, and store the output JWE inside the LUKS header using LUKSMeta.
Here is an example where we bind {\tt /dev/vda2} using the Tang ping:
\begin{lstlisting}[columns=fixed,basicstyle=\ttfamily\footnotesize,tabsize=4,backgroundcolor=\color{yellow!10}]
# clevis bind-luks /dev/sda1 tang '{"url": "http://tang.local"}'
The advertisement is signed with the following keys:
        kWwirxc5PhkFIH0yE28nc-EvjDY

Do you wish to trust the advertisement? [yN] y
Enter existing LUKS password:
\end{lstlisting}

Upon successful completion of this binding process, the disk can be unlocked using one of the unlockers described below.


\paragraph{Dracut}\label{dracut}
The Dracut unlocker attempts to automatically unlock volumes during early boot.
This permits automated root volume encryption.
To unlock our Fedora rebuild initramfs after installing Clevis using:

\begin{lstlisting}[columns=fixed,basicstyle=\ttfamily\footnotesize,tabsize=4,backgroundcolor=\color{yellow!10}]
# dracut -f
\end{lstlisting}
Upon reboot, we will be prompted to unlock the volume using a password.
In the background, Clevis will attempt to unlock the volume automatically.
If it succeeds, the password prompt will be cancelled and boot will continue.
Install it to fedora using:
\begin{lstlisting}[columns=fixed,basicstyle=\ttfamily\footnotesize,tabsize=4,backgroundcolor=\color{yellow!10}]
# dnf install clevis-dracut
\end{lstlisting}
\paragraph{UDisks2}\label{udisk2}
After installation UDisks2 unlocker runs in Fedora desktop session.
There is no need to manually enable it; just install the Clevis UDisks2 unlocker and restart our desktop session.
\begin{lstlisting}[columns=fixed,basicstyle=\ttfamily\footnotesize,tabsize=4,backgroundcolor=\color{yellow!10}]
# dnf install clevis-udisks2
\end{lstlisting}
The unlocker should be started automatically.
This unlocker works almost exactly the same as the Dracut unlocker.
If we insert a removable storage device that has been bound with Clevis, we will attempt to unlock it automatically in parallel with a desktop password prompt.
If automatic unlocking succeeds, the password prompt will be dissmissed without user intervention.
