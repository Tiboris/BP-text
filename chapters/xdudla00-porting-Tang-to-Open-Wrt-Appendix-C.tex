\chapter{LUKS In-Place Encryption}
\label{luksipc}
It takes 4 steps to perform an in place encryption with {\it luksipc} \cite{luksipc}:
\begin{enumerate}
    \item Unmounting the filesystem
    \item Resizing the filesystem to shrink about 10 megabytes (2048 kB is the current LUKS header size -- but do not trust this value, it has changed in the past!)
    \item Performing luksipc
    \item Adding custom keys to the LUKS key-ring
\end{enumerate}



\paragraph{Step 1 -- Unmounting}
There should not be any problems unmounting partition, unless you want to encrypt {\it /} -- the root partition, which in our case (to lock whole disk) will be necessary.
To do so we need to restart our computer and boot any other or live distribution capable of completing these next steps.
\begin{lstlisting}[columns=fixed,basicstyle=\ttfamily\footnotesize,tabsize=4,backgroundcolor=\color{yellow!10}]
# umount /dev/vda2
\end{lstlisting}



\paragraph{Step 2 -- Resizing}
There are plenty tools for re-sizing, essentially for partitioning as whole (fdisk, e2fsck, etc.).
Demonstrating how this is done for ext 2, 3, 4 here:
\begin{lstlisting}[columns=fixed,basicstyle=\ttfamily\footnotesize,tabsize=4,backgroundcolor=\color{yellow!10}]
# e2fsck /dev/vda2
# resize2fs /dev/vda2 -s -10M
\end{lstlisting}
Delete and recreate shrank partition with fdisk:
\begin{lstlisting}[columns=fixed,basicstyle=\ttfamily\footnotesize,tabsize=4,backgroundcolor=\color{yellow!10}]
# fdisk /dev/vda
Welcome to fdisk (util-linux 2.23.2).

Changes will remain in memory only, until you decide to write them.
Be careful before using the write command.

Command (m for help):
\end{lstlisting}
Check the partition number with typing the {\tt p}:
\begin{lstlisting}[columns=fixed,basicstyle=\ttfamily\footnotesize,tabsize=4,backgroundcolor=\color{yellow!10}]
Command (m for help): p
Disk /dev/vda: 407.6 GiB, 437629485056 bytes, 854745088 sectors
Units: sectors of 1 * 512 = 512 bytes
Sector size (logical/physical): 512 bytes / 4096 bytes
I/O size (minimum/optimal): 4096 bytes / 4096 bytes
Disklabel type: dos
Disk identifier: 0x5c873cba
Partition 2 does not start on physical sector boundary.

Device        Boot        Start     End         Blocks      Id    System
/dev/vda1      *          2048      1026047     512000      83    Linux
/dev/vda2                 1026048   1640447     307200      8e    Linux LVM}
\end{lstlisting}



\paragraph{Step 3 -- Encrypting}
After this, luksipc comes into play. It performs an in-place encryption of the data and prepends the partition with a LUKS header. Firt we have to download luksipc or install it with package manager.
\begin{lstlisting}[columns=fixed,basicstyle=\ttfamily\footnotesize,tabsize=4,backgroundcolor=\color{yellow!10}]
$ wget https://github.com/johndoe31415/luksipc/archive/master.zip
$ unzip master.zip
$ cd luksipc-master/
$ make
\end{lstlisting}
Now run it with parameters like:
\begin{lstlisting}[columns=fixed,basicstyle=\ttfamily\footnotesize,tabsize=4,backgroundcolor=\color{yellow!10}]
# ./luksipc -d /dev/vda2
\end{lstlisting}
luksipc will have created a key file /root/initial\_keyfile.bin that you can use to gain access to the newly created LUKS device:
\begin{lstlisting}[columns=fixed,basicstyle=\ttfamily\footnotesize,tabsize=4,backgroundcolor=\color{yellow!10}]
# cryptsetup luksOpen --key-file /root/initial\_keyfile.bin \
    /dev/vda2 fedoradrive
\end{lstlisting}



\paragraph{Step 4 -- Adding key}
DO NOT FORGET to add key to LUKS volume:
\begin{lstlisting}[columns=fixed,basicstyle=\ttfamily\footnotesize,tabsize=4,backgroundcolor=\color{yellow!10}]
# cryptsetup luksAddKey --key-file /root/initial\_keyfile.bin /dev/vda2
\end{lstlisting}
