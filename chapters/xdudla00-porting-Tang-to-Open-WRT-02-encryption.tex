\documentclass[../xdudla00-porting-Tang-to-Open-WRT.tex]{subfiles}
\begin{document}

\chapter{Encryption}\label{encryption}

For most of us is common to have a password protected system.
But the encrypted disk requires another password (key) to decrypt.
Imagine that you came home in mood to enjoy your time and your system ask for password not once but twice.
I think this is reason why most of us do not use encryption even when we know it will protect our data.
This is what Tang \ref{tang} is for and to whom is for because we want automatize things.

Lets look how the encryption is typically done. As you can see in the image below it all starts with desire to keep our data to ourselves and as a secret to the other people.
At the most these secrets are stored on our hard drives. Usually we encrypt this secret by using an encryption key.
But our secret data might grow in size, and it is time and resource consuming to decrypt and encrypt secret every time encryption key changes or it is compromised.
Because of that we wrap encrypted data in the key encryption key and this is what system prompts from us on boot.
So changing key encryption key does not affect encrypted data. We can change it whenever we desire to and redistribute new key to all users who are suposed to access this data.

% does not work on fedora with pdflatex
% \begin{figure}[h]
%     \centering
%     \includegraphics{../figures/placeholder.pdf}
%     \caption{How we encrypt data}
%     \label{fig:encdata}
% \end{figure}

To automatize this we could generate something cryptographicaly stronger than user provided password.
Then we store this cryptographicaly stronger random key on some remote system which is can then get from it.
This is basically how the Escrow \ref{escrow} model works.

\end{document}
