\documentclass[../xdudla00-porting-Tang-to-Open-WRT.tex]{subfiles}
\begin{document}

\chapter{OpenWrt}\label{owrt}

OpenWrt is described as a Linux distribution for embedded devices(typically wireless routers).
It provides a fully writable filesystem with package management, so we are not bound to application provided by the vendor.
It allow us to customize the device through the use of packages to suit any application.

The OpenWrt project started in January 2004.
The first OpenWrt versions were based on Linksys GPL sources for WRT54G and a buildroot from the uClibc project.
Today (January 2017) the stable 15.05.1 release of OpenWrt (codenamed "Chaos Calmer") released in March 2016 using Linux kernel version 3.18.23 runs on hundreds of routers.

\section{OPKG Package manager}

The opkg utility (an ipkg fork) is a lightweight package manager used to download and install OpenWrt packages from local package repositories or ones located in the Internet.
GNU/Linux users already familiar with apt, aptitude, pacman, yum, dnf, etc. will recognize the similarities.
It also has similarities with NSLU2's Optware, also made for embedded devices.
OPKG is however a full package manager for the root file system, instead of just a way to add software to a seperate directory (e.g. /opt).
This also includes the possibility to add kernel modules and drivers.
OPKG is sometimes called Entware, but this is mainly to refer to the Entware repository for embedded devices.

Opkg attempts to resolve dependencies with packages in the repositories - if this fails, it will report an error, and abort the installation of that package. 

Missing dependencies with third-party packages are probably available from the source of the package.
To ignore dependency errors, pass the --force-depends flag.



\end{document}
