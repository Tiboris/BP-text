\chapter{Software portability}\label{porting}

Ideally, any software would be usable on any operating system, platform, and any processor architecture.
Existence of term "porting", derived from the Latin portāre which means "to~carry", proves that this ideal situation does not occurs that often, and acctual process of "carrying" software to system with different environment is most probably required.
Porting is process not documented and required if we want to run thing on another platform.
Porting is also used to describe procces of converting computer games to became platform independent\cite{wiki_porting}.

Software porting procces might be hard to distinguish with building software.
The reason might be that in many cases building software on desired platform is enough, when software application does not work "out of the box".
This kind of behavior, an application that works immediately after or even without any special installation and without need for any configuration or modification, is ideal.
It often happens when we copy application from one computer to another, not realizing that they have processors with same instructions set and same or very similar operating system therefore envoroment.
But let us be realistic, it depends on many things, such as processor architecture to which was application written (compiled), quality of design, how application is meant to be deployed and of course application's source code.

Nowadays, the goal should be to develop software which is portable between preferred computer platforms (Linux, UNIX, Apple, Microsoft).
If software is considered as not portable, it could not have to mean immediately that it is not possible, just that time and resources spent porting already written software are almost comparable, or even significantly higher than writing software as a whole from scratch.
Effort spent porting some software product to work on desired platform must be little, such as copying already installed files to usb flash drive and run it on another computer.
This kind of approach might most probably fail, due to not present dependencies of third party libraries not present on destination computer.
Despite dominance of the x86 architecture there is usually a need to recompile software running, not only on different operating systems, to make sure we have all the dependencies present.

Number of significantly different central processor units (CPUs), and operating systems used on the desktop or server is much smaller than in the past.
However on embeded devices market there are still much more various architectures available including ARM\footnote{https://www.arm.com/products/processors} or MIPS\footnote{https://www.mips.com/products/classic/}.

To simplify portability, even on distinguished processors with distant instruction sets, modern compilers translate source code to a machine-independent intermediate code.
But still, in the embedded system market, where OpenWrt belongs to, porting remains a significant issue \cite{porting_software}.

\section{OpenWrt system}\label{owrt}
OpenWrt is is perhaps the most widely known Linux distribution for embedded devices especially for wireless routers.
It was originally developed in January 2004 for the Linksys WRT54G with buildroot from the uClibc project.
Now it supports many more models of routers.
OpenWrt is a registered trademark which is held by the Software in the Public Interest (SPI) in the name of the OpenWrt project.

Installing OpenWrt system means replacing our router’s built-in firmware with the Linux system which provides a fully writable filesystem with package management.
This means that we are not bound to applications provided by the vendor.
Router (the embedded device) with this distribution can be used for anything that an embedded Linux system can be used for, from using its SSH Server for SSH Tunneling, to running lightweight server software (e.g. IRC server) on it.
In fact it allows us to customize the device through the use of packages to suit any application. \cite{openwrt}

\subsection{OpenWrt and LEDE}

The LEDE Project (“Linux Embedded Development Environment”) is a Linux operating system emerged from the OpenWrt project.
Its announcement was sent on 3th May 2016 by Jo-Philipp Wich to both the OpenWrt development list and the new LEDE development list\footnote{https://lwn.net/Articles/686180/}.
It describes LEDE as "a reboot of the OpenWrt community" and as "a spin-off of the OpenWrt project" seeking to create an embedded-Linux development community "with a strong focus on transparency, collaboration and decentralisation"\footnote{https://www.phoronix.com/scan.php?page=news\_item\&px=OpenWRT-Forked-As-LEDE}.

The rationale given for the reboot was that OpenWrt suffered from longstanding issues that could not be fixed from within—namely, regarding internal processes and policies.
For instance, the announcement said, the number of developers is at an all-time low, but there is no process for on-boarding new developers and, it seems, no process for granting commit access to new developers.

At the moment latest release of OpenWrt 15.05.1 (code-named "Chaos Calmer") released in March 2016.
LEDE developers continued to work separately on their upstream release and they delivered LEDE "Reboot" with version 17.01.0 on February 22nd 2017.

The remerge proposal vote was passed by LEDE developers in June 2017\footnote{http://lists.infradead.org/pipermail/lede-adm/2017-June/000552.html}.
After long and sometimes slowly moving discussions about the specifics of the re-merge, with multiple similar proposals but little subsequent action, formally announced on LEDE forum in January 2018\footnote{https://forum.lede-project.org/t/announcing-the-openwrt-lede-merge/10217}.
OpenWrt and LEDE projects agreed upon their unification under the OpenWrt name.
After merge OpenWrt upstream repository started to show signs of life.

Today (April 2018) the stable LEDE 17.01.4 "Reboot" release of OpenWrt released in October 2017 using Linux kernel version 4.4.92 runs on many routers \cite{lede_release}.

\subsection{Why use OpenWrt}

Custom router firmware may be more stable than our hardware’s default firmware from the vendor.
Not even that but probably more secure with regullar security updates.
Besides OpenWrt there are other open source Linux based firmwares available such as Tomato or DD-WRT.

In the past OpenWrt has supported only CLI (command line interface) configuration therefore it was best match for software developers, network admins, or advanced users.
User not acquainted with Linux or even not comfortable with CLI found the OpenWrt platform hard to use and they turned to other solutions available.

The Tomato firmware is the best match for unexperienced users providing rich GUI and many features specifically live "visual" traffic monitoring, allowing easy visibility on inbound/outbound traffic in real-time.
Big disadvantage of the Tomato is that list of supported devices is quite poor.

DD-WRT on the other hand is compatible with more routers than any other third party firmware.
Compared to Tomato, DD-WRT is reported to have more bugs and less intuitive GUI.
The DD-WRT was known as the most feature rich firmware until OpenWrt came along.

OpenWrt turns our router in a fully capable GNU/Linux computer, not just a network "magic" box.
Last years the OpenWrt platform has come a long way in making itself more accessible to all user levels.
It has an Luci\footnote{https://github.com/openwrt/luci} Web UI(user interface) now.
Users less experienced with Linux can easily set up their network using Luci.
OpenWrt is capable of running lightweight services like an IRC bouncer or samba/ftp file sharing (some routers has USB ports able to power HDD) or even run software build on our own\cite{vpnpick}.

The goal of this thesis would be to port a lightweight Tang daemon described in the chapter \ref{tang} Tang server to OpenWrt system.
Tang server will help us unlock our encrypted volumes while on safe home or office network without need for extra PC running it but having it on our tiny OpenWrt "server".
With Tang we do not have to care about typing passphrases over and over to unlock LUKS drives in safe enviroment.

\section{OpenWrt's toolchain}

Embeded devices are not meant for building on them because they does not have enough memory nor computation resorces as ordinary personal computers do(see Table \ref{routerspec}).
Building on such device would be time consuming and may result to overheating, which could cause hardware to fail.
For this particular reason package building is done with cross-compiler.
Compilation is done by set of tools called toolchain and it consists of:
\begin{itemize}
    \item compiler
    \item linker
    \item a C standard library
\end{itemize}
For porting to "target" system (OpenWrt) this toolchain has to be generated on "host" system.
The toolchain can be achieved in many different ways.
The easiest way is undoubtedly to find a .rpm (or .deb) package and have it installed on our "host" system.
If a binary package with desired toolchain is not available for our system or is not available at all, there might be need to compile a custom toolchain from scratch.\footnote{tools like crosstool-ng (https://github.com/crosstool-ng/crosstool-ng) may help}

In case of OpenWrt we have available a set of Makefiles and patches called buildroot which is capable of generating toolchain.

\subsection{Compiler}

In a nutshell compilers translate the high level programming language source code into lower level language such as machine understandable assembler instructions.
To do so compiler is performing many operations starting with preprocessing.

Preprocessing supports macro substitution and conditional compilation.
Typically the preprocessing phase occurs before syntactic or semantic analysis.
However, some languages such as Scheme\footnote{https://www.scheme.com/tspl4/} support macro substitutions based on syntactic forms.

Lexical analysis, also known as lexing or tokenization, breaks the source code text into a sequence of small pieces called lexical tokens.
Lexical analyzer does the scanning (cutting the input text into tokens and assign them a category) and the evaluating (converting tokens into a processed value).
Common token categories may include:
\begin{itemize}
    \item keywords (bound to programming language)
    \item identifiers (can not be keyword)
    \item separators
    \item operators
    \item literals
    \item comments
\end{itemize}
The set of token categories varies in different programming languages.
The lexeme syntax is typically a regular language, so a finite state automaton constructed from a regular expression can be used to recognize it.

Syntax analysis typically builds a parse tree structure according to the rules of a formal grammar which define the language's syntax.
The parse tree is often analyzed, augmented, and transformed by later phases in the compiler.

Semantic analysis checks parse tree and does type checking, object binding and definite assignment.
Output of semantic analysis is the symbol table or it results into rejecting incorrect programs or issuing warnings.

Code optimization, such as dead code elimination, inline expansion, constant propagation, loop transformation and even automatic parallelization, is done after analysis and it is independent on the CPU architecture.

Code generation is CPU dependent and does the transformation of intermediate language into the target machine language.
This involves resource and storage decisions, such as deciding which variables to fit into registers and memory and generating machine instructions along with their associated addressing modes.

Cross-compiler is programming tool capable of creating executable file that is supposed to run on a "target" architecture, in a similar or completely different enviroment, while working on a different "host" architecture.
It can also create object files used by linker.
Reason for using cross-compiling might be to separate the build environment from target environment as well.
OpenWrt toolchain uses gcc compiler, and it is one of the most important parts of toolchain\cite{compiler}.

\subsection{Linker}

With linker we are able to link compile's object files into single executable, library or object file.
Linker can do static linking and dynamic linking.

Static linking is the result of the linker copying all library routines used in the program into the executable image.
This may require more disk space and memory than dynamic linking, but is more portable, since it does not require the presence of the library on the system where it runs.
Static linking also prevents "DLL Hell", since each program includes exactly the versions of library routines that it requires, with no conflict with other programs. In addition, a program using just a few routines from a library does not require the entire library to be installed.

Dynamic linking is the process of postponing the resolving of some undefined symbols until a program is run.
That means that the executable code still contains undefined symbols, plus a list of objects or libraries that will provide definitions for these.
Loading the program will load these objects/libraries as well, and perform a final linking.
Dynamic linking needs no linker.

Advantage of dynamic linking is that often-used libraries (for example the standard system libraries) need to be stored in only one location, not duplicated in every single binary.
If a bug in a library function is corrected by replacing the library, all programs using it dynamically will benefit from the correction after restarting them.
Programs that included this function by static linking would have to be re-linked first.

Disadvantage known on the Windows platform as "DLL Hell", an incompatible updated library will break executables that depended on the behavior of the previous version of the library if the newer version is not properly backward compatible\cite{compiler}.
\subsection{C standard library}

Most common C standard libraries are: GNU Libc, uClibc musl-libc, or dietlibc.
They provide macros, type definitions and functions for tasks such as input/output processing (<stdio.h>), memory management (<stdlib.h), string handling (<string.h>), mathematical computations (<math.h>) and many more\footnote{https://en.wikipedia.org/wiki/C\_standard\_library\#Header\_files}.
The OpenWrt's cross-compilation toolchain uses musl-libc \cite{c_library}.

\section{OpenWrt's buildroot}

OpenWrt's buildroot is a build system capable of generating toolchain, and also a root filesystem (also called sysroot), enviroment tightly bound to target.
Build system can be configured for any device that is supported by OpenWrt.

The root filesystem in general is a mere copy of the file system of target's platform.
In many cases just having the folders /usr and /lib would be sufficient, therefore we do not need to copy nor create the entire target file system on our host .

It is a good idea to store all these things, the toolchain and the root filesystem in a single place.
With using OpenWrt's buildroot we will have this covered.
Be tidy and pedantic, because cross-compiling can easily become a painful mess!\cite{fabrizio}

\subsection{Buildroot prerequisites}

Let us demonstrate what are minimum requirements of space and size of RAM (Random Access Memory) for building packages for Openwrt using its buildroot.
For generating an installable OpenWrt firmware image file with a size of e.g. 8MB, we will need at least:
\begin{itemize}
\item ca. 200 MB for OpenWrt build system
\item ca. 300 MB for OpenWrt build system with OpenWrt Feeds
\item ca. 2.1 GB for source packages downloaded during build from the Feeds
\item ca. 3-4 GB to build (i.e. cross-compile) OpenWrt and generate the firmware file
\item ca. 1-4 GB of RAM to build Openwrt.(build x86's img need 4GB RAM)
\end{itemize}

These are specifications of embedded device TL-WR842Nv3 a regular wireless router manufactured by TP-LINK which was used to test all packages related to Tang server porting effort:

\begin{table}[h]
\centering
\label{routerspec}
\begin{tabular}{c|c}
\hline
Model           &   TL-WR842N(EU)                   \\ \hline
Version         &   v3                              \\ \hline
Architecture:   &   MIPS 24Kc                       \\ \hline
Manufacturer:   &   Qualcomm Atheros                \\ \hline
Bootloader:     &   U-Boot                          \\ \hline
System-On-Chip: &   Qualcomm Atheros QCA9531-BL3A   \\ \hline
CPU Speed:      &   650 MHz                         \\ \hline
Flash chip:     &   Winbond 25Q128CS16              \\ \hline
Flash size:     &   16 MiB                          \\ \hline
RAM chip:       &   Zentel A3R12E40CBF-8E           \\ \hline
RAM size:       &   64 MiB                          \\ \hline
Wireless:       &   Qualcomm Atheros QCA9531        \\ \hline
Antenae(s):     &   2 non-removable                 \\ \hline
Ethernet:       &   4 LAN, 1 WAN 10/100             \\ \hline
USB:            &   1 x 2.0                         \\ \hline
Serial:         &   ?                               \\ \hline
\end{tabular}
\caption{TL-WR842Nv3 specifications}
\end{table}

The actual sum of space required only for buildroot to work correctly which is about 6.4 GB.
Comparation to available storage space on wireless router should demonstrate why is building for embedded devices done with cross-compiling.
The another reasons, not to just compare internal storage which might be extended\footnote{https://wiki.openwrt.org/doc/howto/extroot} are device's minimalistic RAM size and lower computation capability of the CPU.

\section{Preparing the host enviroment}

To start with cross-compilation on the host system we need to set up an enviroment for it.
As the OpenWrt buildroot is set of scripts it has runtime dependencies.
We need to install these dependencies first.
To install buildroot dependencies on Fedora 27 system run:
\begin{lstlisting}[columns=fixed,basicstyle=\ttfamily\footnotesize,tabsize=4,backgroundcolor=\color{yellow!10}]
# dnf install binutils bzip2 gcc gcc-c++ \
gawk gettext git-core flex ncurses-devel ncurses-compat-libs \
zlib-devel zlib-static make patch perl-ExtUtils-MakeMaker \
perl-Thread-Queue  glibc glibc-devel glibc-static quilt sed \
sdcc intltool sharutils bison wget unzip openssl-devel
\end{lstlisting}
Some feeds (the OpenWrt packages) might be not available over git but only via other versioning tools like svn (subversion) or mercurial.
If we want to obtain their source-code, we need to install svn and mercurial as well:
\begin{lstlisting}[columns=fixed,basicstyle=\ttfamily\footnotesize,tabsize=4,backgroundcolor=\color{yellow!10}]
# dnf install subversion mercurial
\end{lstlisting}
Please note that these commands have to be run by user with root privileges.

\subsection{Getting buildroot}

OpenWrt system has available an SDK(Software development kit) buildroots for every released version of OpenWrt system.
It is good to consider using OpenWrt's SDK in order to build software application to only one specific release of target system.
For example when we are using "stable" release of OpenWrt 15.05.1 with codename "Chaos Calmer" on TL-WR842N(EU) device, we should probably end up in in "Supplementary Files" section of the OpenWrt archives
\footnote{https://archive.openwrt.org/chaos\_calmer/15.05.1/ar71xx/generic/} looking for the SDK.
While porting an upstream projets (such as Tang) for latest target release a bleeding edge buildroot would be the best solution.

If we have host platform which aims to be on the bleeding edge of open-source technologies, such as OS Fedora, we will probably encounter issues with dependencies.
These issues are apearing because an older version of the dependencies are required for this old SDK to work and might be not available for our host platform anymore.
In this case we would have to install or even build older version of required packages.

We will work with bleeding edge (trunk version) buildroot and build for latest LEDE 17.01.4 release.
To clone upstream buildroot run command:
\begin{lstlisting}[columns=fixed,basicstyle=\ttfamily\footnotesize,tabsize=4,backgroundcolor=\color{yellow!10}]
$ git clone https://github.com/openwrt/openwrt.git
\end{lstlisting}

\section{Working with buildroot}\label{working-with-buildroot}

Working with Openwrt's buildroot requires some basic knowledge of the git version control system\footnote{https://git-scm.com/docs/gittutorial} and processes of upstream project development using GitHub\footnote{https://guides.github.com/introduction/flow/}.
Short description how to fork and set up repository can be found on appendix \ref{set_repo} Setting up the repository.
All work related to this thesis has been done using GitHub account Tiboris\footnote{https://github.com/Tiboris/} therefore following command to get our fork of buildroot would be:
\begin{lstlisting}[columns=fixed,basicstyle=\ttfamily\footnotesize,tabsize=4,backgroundcolor=\color{yellow!10}]
$ git clone https://github.com/Tiboris/openwrt.git ~/buildroot-openwrt
\end{lstlisting}

After we have enviroment ready to be worked on (all needed dependencies installed and fork of our buildroot downloaded on host system) let us just follow these simple 4 rules to not break our enviroment in any way.
\begin{enumerate}
    \item Do everything in buildroot as non-root user!
    \item Issue all OpenWrt build system commands in the <buildroot> directory, \\e.g. $\sim$/buildroot-openwrt/
    \item Do not build in a directory that has spaces in its full path.
    \item Change ownership of the directory where we downloaded \\the OpenWrt to other than root user
\end{enumerate}
Having these in mind will definitely make our work easier.

\subsection{Setting feeds}

In OpenWrt, a "feed" is a collection of packages which share a common location.
Feeds may reside on a remote server, in a version control system, on the local filesystem, or in any other location addressable by a single name (path/URL) over a protocol with a supported feed method.
It is the most important step to do before starting cross-compilation.
As the listing \ref{default_feeds} Content of feeds.conf.default shows a list of usable feeds is configured from the feeds.conf file or feeds.conf.default when feeds.conf does not exist.

\begin{lstlisting}[columns=fixed,basicstyle=\ttfamily\footnotesize,tabsize=4,label=default_feeds,caption=Content of feeds.conf.default]
    src-git packages https://git.openwrt.org/feed/packages.git
    src-git luci https://git.openwrt.org/project/luci.git
    src-git routing https://git.openwrt.org/feed/routing.git
    src-git telephony https://git.openwrt.org/feed/telephony.git
    #src-git video https://github.com/openwrt/video.git
    #src-git targets https://github.com/openwrt/targets.git
    #src-git management https://github.com/openwrt-management/packages.git
    #src-git oldpackages http://git.openwrt.org/packages.git
    #src-link custom /usr/src/openwrt/custom-feed
\end{lstlisting}

The new custom OpenWrt packages are located in packages feed (see the first line of the \ref{default_feeds} Content of feeds.conf.default listing).
To work with our own fork of the packages we can simply change the line to point to our fork repository.
Feeds can point to special branch of our choice:
\begin{lstlisting}[columns=fixed,basicstyle=\ttfamily\footnotesize,tabsize=4,backgroundcolor=\color{yellow!10}]
src-git packages https://github.com/Tiboris/packages-OpenWrt.git;new_packages
\end{lstlisting}
or commit hash in repository:
\begin{lstlisting}[columns=fixed,basicstyle=\ttfamily\footnotesize,tabsize=4,backgroundcolor=\color{yellow!10}]
src-git packages https://github.com/Tiboris/packages-OpenWrt.git^dbdfc99
\end{lstlisting}
Changing this feed to will reduce effort spending on getting all new dependencies to the buildroot's feeds/packages directory.

The feeds are "managed" with the script available in builroot's scripts directory, {\it feeds}.
To download feeds run this script with command {\it update} and option {\it -a}.
Remember to invoke it from the buildroot directory($\sim$/buildroot-openwrt).
\begin{lstlisting}[columns=fixed,basicstyle=\ttfamily\footnotesize,tabsize=4,backgroundcolor=\color{yellow!10}]
$ ./scripts/feeds update -a
\end{lstlisting}

To make any feed available for build we shall "install" it using the same feeds script:
\begin{lstlisting}[columns=fixed,basicstyle=\ttfamily\footnotesize,tabsize=4,backgroundcolor=\color{yellow!10}]
$ ./scripts/feeds install <PACKAGENAME>
\end{lstlisting}
or we can use option {\it -a} instead of the {\it <PACKAGENAME>} and make all of the feeds available for build.

If for some unknown reason the issued update of packages does not seems to show all the updates it might be helpful to clean the buildroot's {\it tmp/} directory\cite{feeds}.
\begin{lstlisting}[columns=fixed,basicstyle=\ttfamily\footnotesize,tabsize=4,backgroundcolor=\color{yellow!10}]
$ rm -rf tmp/
\end{lstlisting}

\subsection{The menuconfig}

For OpenWrt it has been the intention from the beginning, with the development of menuconfig, to create a simple, yet powerful, environment for the configuration of individual builds.
Working with menuconfig is very intuitive and the tool is more or less self-explanatory.
Even the most specialized configuration requirements can be met by using it.
Depending on the particular target platform, package requirements and kernel module needs, the standard configuration process will include modifying:
\begin{itemize}
    \item Target system
    \item Package selection
    \item Build system settings
    \item Kernel modules
\end{itemize}

Start the menuconfig interface shown on Figure \ref{fig_menuconfig} menuconfig by issuing the following command:
\begin{lstlisting}[columns=fixed,basicstyle=\ttfamily\footnotesize,tabsize=4,backgroundcolor=\color{yellow!10}]
$ make menuconfig
\end{lstlisting}
The make menuconfig will collect package info first.
Afterwards the following window will appear in our terminal and there we can start configuring the image using menuconfig:
\begin{figure}[h]
    \centering
    \includegraphics[scale=0.6]{figures/make_menuconfig.pdf}
    \caption{menuconfig}
    \label{fig_menuconfig}
\end{figure}

For configuring the image we have three options: y, m, n which are represented as follows:
\begin{itemize}
\item pressing y sets the <*> built-in label\\
This package will be compiled and included in the firmware image file.
\item pressing m sets the <M> package label\\
This package will be compiled, but not included in the firmware image file. (E.g. to be installed with opkg after flashing the firmware image file to the device.)
\item pressing n sets the < > excluded label\\
The source code will not be processed.
\end{itemize}
When we save our configuration, the file .config will be created in the buildroot directory according to desired configuration.

Target system is selected from the extensive list of supported platforms, with the numerous target profiles – ranging from specific devices to generic profiles, all depending on the particular device at hand.
In our case we should browse the "Target Profile" selection and find our targeted device (TP-LINK TL-WR842N/ND v3).

Package selection has the option of either 'selecting all package', which might be un-practical in certain situation, or relying on the default set of packages will be adequate or make an individual selection.
It is here needed to mention that some package combinations might break the build process, so it can take some experimentation before the expected result is reached.
Added to this, the OpenWrt developers are themselves only maintaining a smaller set of packages – which includes all default packages – but, the feeds-script makes it very simple to handle a locally maintained set of packages and integrate them in the build-process.

The final step before the process of compiling the intended image would be to exit the menuconfig tool – this also includes the option to save a specific configuration or load an already existing, and pre-configured, version.
Exit the UI, and choose to save the settings\cite{build_owrt}.

\subsection{Building single Packages}

When developing or packaging software for OpenWrt, it is convenient to be able to build only the package in question (e.g. with package cups):
\begin{lstlisting}[columns=fixed,basicstyle=\ttfamily\footnotesize,tabsize=4,backgroundcolor=\color{yellow!10}]
$ make package/cups/compile V=s
\end{lstlisting}
For a rebuild run:
\begin{lstlisting}[columns=fixed,basicstyle=\ttfamily\footnotesize,tabsize=4,backgroundcolor=\color{yellow!10}]
$ make package/cups/{clean,compile,install} V=s
\end{lstlisting}
It doesn't matter what feed the package is located in, this same syntax works for any installed package.

If for some reason the build fails, the easiest way to spot the error is to do:
\begin{lstlisting}[columns=fixed,basicstyle=\ttfamily\footnotesize,tabsize=4,backgroundcolor=\color{yellow!10}]
$ make V=s 2>&1 | tee build.log | grep -i error
\end{lstlisting}
Complete guide can be found on OpenWrt Wiki\footnote{https://wiki.openwrt.org/doc/howto/build}.
