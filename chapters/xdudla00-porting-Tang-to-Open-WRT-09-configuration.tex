\chapter{Tang configuration}\label{config}

\section{OPKG Package manager}

The opkg utility (Open Package Management System) is a lightweight package manager used to download and install OpenWrt packages.
The opkg is fork of an ipkg (Itsy Package Management System).
These packages could be stored somewhere on device's filesystem or the package manager will download them from local package repositories or ones located on the Internet mirrors.
Users already familiar with GNU/Linux package managers like apt/apt-get, pacman, yum, dnf, emerge etc. will definitely recognize the similarities.
It also has similarities with NSLU2's Optware, also made for embedded devices.
Fact that OPKG is also a full package manager for the root file system, instead of just a way to add software to a seperate directory (e.g. /opt) includes the possibility to add kernel modules and drivers.
OPKG is sometimes called Entware, but this is mainly to refer to the Entware repository for embedded devices.

Opkg attempts to resolve dependencies with packages in the repositories - if this fails, it will report an error, and abort the installation of that package.

Missing dependencies with third-party packages are probably available from the source of the package.
