\documentclass[../xdudla00-porting-Tang-to-Open-WRT.tex]{subfiles}
\begin{document}

\chapter{Escrow}\label{escrow}
Before Tang, automated decryption was handled usually with Key Escrow \cite{escrow} (also known as a “fair” cryptosystem).
Client using Key Escrow usually generates a key, encrypt data with it and then stores the key encryption key in a remote server.
But it is not as simple as it sounds.

To deliver those keys we desire to store on Escrow server we have got to encrypt the channel on which we distribute the key.
Without encrypted link when transmitting keys over not secure network anyone who is listening to the network traffic could immadiately fetch this key.
This should signal security risks and of course we do not want third party to acces our secret data.
Usually we encrypt channel with TLS \cite{tls}(Transport Layer Security) or GSSAPI \cite{gssapi}
 (Generic Security Services Application Program Interface) as you can see on Figure \ref{fig:escrowmodel} below.
Unfortunatelly this is not enough to call the communication secure.

We can not just start sending these keys to server if we do not know that this server is one that it act to be.
This server has to have its own identity to trust and we have to authenticate to this server too.
Increasing amount of keys implicates need for Certification Authority server (CA)\cite{ca} or Key Distribution Center (KDC) \cite{kdc} to manage all of them.
With all these keys and at this point only, server can verify if person is permitted to get his or her key, and client is able to identify trusted server.
This is a fully statefull process.
To sum up an authorized third party may gain access to keys stored on Escrow server under certain circumstances only.

\begin{figure}[h]
    \centering
    %\includegraphics{../figures/placeholder.pdf}
    \caption{Escrow model}
    \label{fig:escrowmodel}
\end{figure}

Complexity of this system increases the attack surface and for this complex system would be unimaginable to not have backups.
Escrow server may store lots of keys from lots of different places and basically we can not afford to loose them.

\end{document}
