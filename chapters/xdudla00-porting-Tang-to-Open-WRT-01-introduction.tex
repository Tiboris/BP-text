\chapter{Introduction}\label{introduction}
\epigraph{\it We spend our time searching for security, and hate it when we get it.}{{John Steinbeck}\cite{quote}}

Nowadays, the whole world uses information technologies to communicate and to spread knowledge in form of bits to the other people.
But there are pieces of personal information such as photos from family vacation, videos of our children as they grow, contracts, testaments which we would like to protect.

{\it Encryption}, as described in chapter \ref{encryption} How we use Encryption, protects our data and privacy even when we do not realize that.
It provides process of transforming our information in such way that only trusted person or device can decrypt data and retrieve it.
An unauthorized party might be able to access secured data but will not be able to read the information from it without the proper key.
The most important thing is keeping the encryption key a secret.

With an increasing number of encryption keys to store and protect, there might be necessary to consider using Key management server.
This server should provide a secure and persistent service to its clients.
One of the possible solutions for persistent Key management is to deploy {\it Key Escrow} server described in section \ref{escrow}.
Another solution is server {\it Tang}, which principles are mentioned in chapter \ref{tang} Tang server.
Tang is completely anonymous key recovery service.
In contrast to Key Escrow server, Tang does not know any key.
It only provides mathematical operation for its clients to recover them.

The goal of this bachelor thesis is to port the {\it Tang} server, and its dependencies listed in section \ref{dependencies} to the OpenWrt system.
{\it OpenWrt}, characterized in chapter \ref{owrt} OpenWrt system, is Linux based operating system for {\it embedded} devices such as Wireless routers.

To achieve this goal \dots \todo{thesis tour correct refs}

With accomplishing thesis goal, we will be able to automatize process of unlocking encrypted drives on our private home or small office network, therefore securing our data stored on personal computer's hard drive and/or NAS (Network-attached storage) server if it is stolen.
There will be no need for any decryption or even Key Escrow server but the {\it OpenWrt} device running the {\it Tang} server itself only.
