\documentclass[../xdudla00-porting-Tang-to-Open-WRT.tex]{subfiles}

%\usepackage{epigraph}

\begin{document}

\chapter{Introduction}\label{introduction}
\epigraph{\it We spend our time searching for security and hate it when we get it.}{\textit{John Steinbeck}\cite{quote}}

Nowadays, the whole world uses information technologies to communicate and to spread knowledge in form of bits to the other people.
But there are personal information such as photos from family vacation, videos of our children as they grow, contracts, testaments, and so on which we would like to protect.

Encryption protects our data and privacy even when we do not realize that.
It provides process of transforming our information in such way that only trusted person or device can decrypt data and access it.
An unauthorized party might be able to access secured data but will not be able to read the information from them without the proper key.
The most important thing is keeping the encryption key a secret.

The goal of this bachelor thesis will be to port {\it Tang} server \ref{tang} and its dependencies \ref{dependencies} to {\it OpenWrt} system \ref{owrt}.
With accomplishing this, we will be able to automatize process of unlocking encrypted drives on our private home network.
There will be no need for any decryption server but only {\it OpenWrt} device running the {\it Tang} server itself.

\end{document}
