\documentclass[../xdudla00-porting-Tang-to-Open-WRT.tex]{subfiles}
\begin{document}
%=========================================================================
% (c) Michal Bidlo, Bohuslav Křena, 2008

\chapter{Introduction}\label{introduction}

Nowadays, the whole world uses information technologies to communicate and simply to spread knowledge in form of bits to the other people.  But there are personal information such as photos from a family vacation, videos of our children as they grow, contracts, testaments and so on which we would like to protect.

Encryption protects our data even when we do not know that. It protects our conversation privacy, personal information stored all over governmental authorities, bank accounts, in general data transmitted around the internet and stored on our hard drives. Encryption provides process of transforming our information in such way that only trusted person can decrypt data and access it. An unauthorized person might be able to access secured data but will not be able to read the information from them without the proper key. The most important thing is keeping the encryption key (the password) a secret.

The goal of this bachelor thesis will be to port % https://en.wikipedia.org/wiki/Porting
{\it Tang} server \ref{tang}, its dependencies, and {\it Clevis} framework \ref{clevis} to {\it OpenWrt} system \ref{owrt}. With accomplishing, this we will be able to automatize process of unlocking encrypted drives on our private respectively home network. There will be no need for any decryption server but only {\it OpenWrt} device running the {\it Tang} server itself.

%My primary goal is to focus on the packaging system of the {\it OpenWrt} and build processes on it.\todo{Treba to doplnit...}

%($\backslash${\tt cite\{identifikator\}}).


\end{document}
