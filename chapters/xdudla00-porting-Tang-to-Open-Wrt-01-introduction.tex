\chapter{Introduction}\label{introduction}
\epigraph{\it We spend our time searching for security, and hate it when we get it.}{{John Steinbeck}\cite{quote}}

Nowadays, the whole world uses information technologies to communicate and to spread knowledge in form of bits to the other people.
But there are pieces of personal information such as photos from family vacation, videos of our children as they grow, contracts or testaments which we would like to protect.

{\it Encryption}, as described in chapter \ref{encryption}, protects our data and privacy even when we do not realize that.
An unauthorized party may be able to access secured data but will not be able to read the information from it without the proper key.
With an increasing number of encryption keys to store and protect, it might be necessary to consider using Key management server.
One of the possible solutions for persistent Key management is to deploy {\it key escrow} server described in section \ref{escrow}.
Another solution is server {\it Tang}, whose principles are mentioned in section \ref{tang}.

Tang is completely anonymous key recovery service aiming to solve early boot decryption of system volumes encrypted with LUKS specification, described in section \ref{LUKS_disk}.
In contrast to Key Escrow server, Tang does not know any keys.
It only provides mathematical operation for its clients to recover them.

The goal of this bachelor thesis is to port and document this process of porting {\it Tang} server to the OpenWrt system.
{\it OpenWrt}, characterized in section \ref{owrt}, is Linux-based operating system for {\it embedded} devices such as wireless routers.
Porting packages for the OpenWrt as described in chapter \ref{porting} is done with cross-compilation tools available from OpenWrt's buildroot.

The process of porting missing dependencies for the Tang server is described in chapter \ref{satisfy_dependencies}.
Work required for the Tang itself is divided into chapter \ref{porting-tang} and chapter \ref{config}.
Section \ref{limitations} sums up not only the limitations that are present on OpenWrt platform but also generic limitation that was discovered while testing the solution.

Result presented in this thesis allows us to automate process of unlocking encrypted drives on our private home or small office network, therefore securing data stored on personal computer's hard drive and/or NAS (Network-attached storage) server if it is stolen.
There will be no need for any decryption or even Key Escrow server except the {\it OpenWrt} device running the {\it Tang} server.
