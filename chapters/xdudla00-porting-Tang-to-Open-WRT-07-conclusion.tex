\documentclass[../xdudla00-porting-Tang-to-Open-WRT.tex]{subfiles}
\begin{document}


\chapter{Conclusion}\label{conlusion}

The Tang \ref{tang} server is very lightweight program.
It provides secure and anonymous data binding using McCallum-Relyea exchange \ref{fig:mrexchange} algorythm.

Clevis \ref{clevis} is a client software with full support for Tang.
It has minimal dependencies and it is possible to use with HTTP, Escrow \ref{escrow}, and it implements Shamir Secret Sharing\cite{sss}.
Clevis has GNOME \cite{gnome} integration so it is not only a command line tool.
Clevis also supports early boot integration with dracut\ref{dracut} or even removable devices unocking using UDisks2 \ref{udisk2}.

To port Tang to OpenWrt system \ref{owrt} will be necesarry to port all its dependencies first.
The OpenWrt system has already package openssl \ref{openssl}, zlib \ref{zlib}, and jansson \ref{jansson} but only version 2.7 which is too old.
So there will be need for porting jansson. José will require porting and http-parser \ref{parser} too.
The systemd \ref{systemd} would be huge effort but tang's requirements are minimal and we should be able to
work with xinetd \cite{xinetd}. Finally porting Tang itself.

\end{document}
