\documentclass[../xdudla00-porting-Tang-to-Open-WRT.tex]{subfiles}
\begin{document}

\chapter{Ukazka rozsireni sablony}

\section{Definice v projekt.tex}

Kazde nastaveni sablny se hodi pro neco jineho. Minamlne pouzijete tyto:
\begin{itemize}
    \item \begin{verbatim}\documentclass[]{fitthesis}\end{verbatim}
    \item \begin{verbatim}\documentclass[zadani,print]{fitthesis}\end{verbatim}
\end{itemize}

Experimentujte a vyzkousejte i zbyle moznosti napr s ustavnim logem, seznamem obrazku, tabulek, atd\dots

\begin{verbatim}
\documentclass[]{fitthesis} % bez zadání - pro začátek práce, aby nebyl
    problém s překladem
%\documentclass[english]{fitthesis} % without assignment - for the work
    start to avoid compilation problem
%\documentclass[zadani]{fitthesis} % odevzdani do wisu - odkazy jsou barevné
%\documentclass[english,zadani]{fitthesis} % for submission to the
    IS FIT - links are color
%\documentclass[zadani,print]{fitthesis} % pro tisk - odkazy jsou černé
%\documentclass[english,zadani,print]{fitthesis} % for the print - links
are black
%\documentclass[listFigures,listTables,listAppendices]{fitthesis}
% s listem obrazku, tabulek a priloh
%\documentclass[uifs]{fitthesis} % ustavni logo
\end{verbatim}

\newpage
\section{Ukazky kodu}
Ukazka kodu pro Microsoft Network monitor v jazyce NPL \autoref{nplExample}.

\begin{lstlisting}[style=npl,label=nplExample,caption=Example of not complete NPL description of YMSG IM protocol. Commented parts are not supported by NPlangCompiler.]
//[RegisterAfter(TCPPayload.HTTP, YMSG, 5050)]
//[RegisterAfter(PayloadHeader.LLC, YMSG, YMSG)]
Protocol YMSG = FormatString("Type = %s (%d), status = %s (%d)",
                     YMSGTypes(Type),Type, YMSGHEADStatus(Status),Status)
{
  AsciiString(4) YMSGconst;
  UINT16 Version;
  UINT16 VendorID;
  [Post.Properties.Length = Length]
  UINT16 Length;
  UINT16 Type = FormatString("%s (%#04x)", YMSGTypes(Type), Type);
  UINT32 Status = FormatString("%s (%#04x)", YMSGHEADStatus(this), this);
  UINT32 Session;

/* [BuildConversationWithParent(Session)]	StartPayload
  [YMSGPayload = Blob(FrameData, FrameOffset, Length),
  DataFieldFrameLength = frameOffset + Length,
  PayloadStart (
  NetworkDirection, //direction
  0, // id
  0, // sequence token
  0, // next sequence token
  Length, // total payload length
  !Property.TCPContinuation, // is first
  (TCP.Flags.Push || TCP.Flags.Fin || TCP.Flags.Urgent),//is last
  RssmblyIndStartBit + RssmblyIndEndBit // Properties...   )]*/

  [HeaderOffset = FrameOffset]
  TVs tvs;
  status = "%s (%d)", YMSGTypes(Type),Type,YMSGHEADStatus(Status),Status)]
  switch{
    case (FrameLength - FrameOffset - 1 > 0) && (HeaderOffset <= FrameOffset):
      [MultiYMSG = "YES"]
      StringTerm(0, "YMSG", 1, 0, 0 ) blank0 = FormatString("Optional stuffing between multiple YMSG messages", ymsg);
      YMSG ymsg;
    case (FrameLength - FrameOffset > 0) && (HeaderOffset <= FrameOffset):
      BLOB(FrameLength - FrameOffset) blank1 = FormatString("Optional terminator of YMSG message", this);
  }
}
\end{lstlisting}

\newpage
Ukazka XML vstupu \dots, \autoref{xmlLogIM}.
\begin{lstlisting}[language=XML,basicstyle=\ttfamily\footnotesize,label=xmlLogIM,caption=Example of IMSleuths`s xml log used as input in ContextBrowser.]
<?xml version="1.0" encoding="utf-8"?>
<log>
  <user guid="192.168.2.101">
    <protocol val="ICQ">
      <event validity="valid">
        <src version="IPv4" port="49874">192.168.2.101</src>
        <dst version="IPv4" port="5190">205.188.10.251</dst>
        <conversationPart type="statusChange" timeStamp="2013-08-08T02:06:19.518767"
        frameNumber="86" flowDirection="up" statusType="online" />
        <conversationPart type="contactList" timeStamp="2013-08-08T02:06:39.318927"
        frameNumber="113" flowDirection="down">
          <group id="0" name="">
            <contact id="3" firstName="277264821"
              nick="Contact1" />
          </group>
          <group id="1" name="rename_group" />
          <group id="3" name="friends_group">
            <contact id="9327" firstName="647175775"
              nick="Contact2" />
            <contact id="2" firstName="284569266"
              nick="Contact3" />
            <contact id="29425" firstName="312345170"
              nick="Contact4" />
          </group>
        </conversationPart>
        <conversationPart type="message"
        timeStamp="2013-08-08T02:11:18.357794" frameNumber="791" flowDirection="up">
          <![CDATA[<HTML><BODY dir="ltr"><FONT size="2">test</FONT></BODY></HTML>]]>
          <receiver>310451170</receiver>
        </conversationPart>
        <conversationPart type="contactListChange"
        timeStamp="2013-08-08T02:32:11.914733" frameNumber="1528" flowDirection="up"
        contactListChangeAction="add">
          <contact id="31432670" />
        </conversationPart>
        <conversationPart type="authorizationReply"
        timeStamp="2013-08-08T02:32:55.885157"
          frameNumber="1543" flowDirection="down" sender="310451170"
          authorizationStatus="permit" />
    </protocol>
  </user>
</log>
\end{lstlisting}

\begin{lstlisting}[language=c,basicstyle=\ttfamily\footnotesize,label=c,caption=Basic C code.]
    #include<stdio.h>
    #include<iostream>
    int main(void)
    {
    printf("Hello World\n");
    // comment
    return 0;
    }
\end{lstlisting}

\section{TODO}

Kdyz pouzijete TODO \todo{Neco bych mel doplnit...}.

\end{document}
