Hlavným cieľom tejto práce je naportovať a zdokumentovať tento proces sprístupnenia serveru Tang, na vstavané zariadenia typu WiFi smerovač, s~plne modulárnym operačným systémom OpenWrt.
Tým dosiahneme anonymnú správu šifrovacích kľúčov pre domáce siete a siete malých firiem.
Dosiahnutie hlavného cieľa je možné vďaka procesu portovania a krížovej kompilácie na platforme Linux.

Táto práca popisuje problematiku šifrovania a jeho využitie na zabezpečenie pevného disku počítača.
Oboznamuje čitateľa so štruktúrou šifrovaného diskového oddielu podľa LUKS špecifikácie na operačných systémoch typu Linux.
Práca rozoberá možnosti {au\-to\-ma\-ti\-zá\-cie} odomykania šifrovaných diskov použitím externého servera {\it Key Escrow} alebo {\it Tang}, ktorý vstupuje do procesu ako tretia strana.

Server Key Escrow slúži na ukladanie šifrovacích kľúčov.
Na rozdiel od serveru Tang, Key Escrow potrebuje zabezpečenú infraštruktúru a identifikovať každého používateľa, ktorý chce jeho služby využívať.
Server Tang poskytuje anonymnú obnovu kľúčov, a teda žiaden z~nich nepozná, iba poskytne matematickú operáciu klientovi, ktorý žiada o~obnovenie.
Operácie, ktoré server Tang vykonáva pri tejto obnove sú detailne opísané v~práci.

Krížovú kompiláciu vykonáva upravený kompilátor, ktorý generuje binárny kód s mož\-nos\-ťou spustenia na inej platforme, než je tá, na ktorej je spustený samotný preklad zdrojových kódov.
Používa sa v~prípadoch, keď sú zdrojové kódy spoločné pre viacero cieľových platforiem, na ktorých môže byť program spustený (Linux a Microsoft Windows) alebo rovnaký systém s~rôznymi typmi procesorov (16bitový, 32bitový a 64bitový) alebo architektúr (x86\_64, MIPS, ARM).
Často sa využíva na generovanie spustiteľných súborov pre vstavané systémy a pri preklade na platformy, ktoré nie sú schopné kompilácie.

Tento kompilátor je spolu s~linkerom a štandardnou knižnicou jazyka C hlavnou súčasťou súboru počítačových programov, nástrojov alebo utilít, nazývaným aj tool-chain.
\mbox{Tool-chain} pre systém OpenWrt je možné vygenerovať použitím buildroot-u z~GitHub repozitára \mbox{openwrt/openwrt}.
Tento buildroot obsahuje pomocné skripty a Makefile-y uľahčujúce automatizáciu procesu vytvárania inštalovateľného obrazu systému OpenWrt.
Buildroot je možné nakonfigurovať pre špecifické zariadenie podporované systémom OpenWrt.

Okrem schopnosti vytvárania obrazov systému OpenWrt je tento nástroj používaný na vytváranie inštalovateľných softvérových balíkov pre túto platformu.
Pre portnutie, teda vytvorenie balíka, je potrebné vytvoriť adresárovú štruktúru v~repozitári feeds.
Táto adresárová štruktúra obsahuje súbory zdrojového kódu pre rôzne softvérové projekty spolu s~Makefile-om, ktorý obsahuje direktívy buildroot-u pre spustenie krížovej kompilácie na cielenú architektúru.
Takto vytvorené balíky je možné následne nainštalovať na zariadenia s~nainštalovaným systémom OpenWrt.

Počas procesu balenia softvéru pomocou buildroot-u OpenWrt môže dôjsť k~mnohým chybám.
Najčastejšie sú chýbajúce závislosti knižníc pre úspešné skompilovanie a prepínače pre linker a kompilátor v~Makefile pre OpenWrt.
Tieto chyby sa riešia postupne, ako sa počas kompilácie v~buildroot-u vyskytnú, preto ich riešenie je aj časovo náročnejšie.

Pre úspešné skompilovanie softwéru Tang bolo potrebné zaistiť, aby všetky jeho závislosti boli dostupné pre OpenWrt.
Pre získanie informácií o~týchto softvérových závislostiach a ich verziách sme využili Internetové stránky projektu {\it Tang} a {\it koji} obsahujúcej informácie o~balíčkoch pre distribúciu Fedora.

Niektoré softvérové závislosti už boli dostupné pre OpenWrt systém v~jeho repozitároch.
Závislosť jansson potrebovala aktualizovať na novšiu verziu, čo sa nám aj podarilo.
Komunita náš pull-request prijala takmer okamžite.
Knižnica libhttp-parser bola zdrojom viacerých problémov, ktoré sa nám ale poradilo vyriešiť ako pre Tang, tak aj pri aktualizovaní verzií a komunikácií s~komunitou OpenWrt.

Závislosť knižnice a nástroja José si vyžadovala vytvorenie nového balíka, ktorý nebol dostupný pre zariadenia OpenWrt.
Túto závislosť sa nám podarilo portnúť a po komunikácii s~komunitou na GitHub-e je balík pripravený na posudok.

Po odstránení závislosti systemd pre Tang a problémov s~knižnicou http-parser sa nám nakoniec podarilo úspešne portnúť, a teda aj zabaliť softvér serveru Tang do balíka in\-šta\-lovateľného na platforme OpenWrt.
Tento balík sme niekoľko krát upravovali a zariadili tak to, že používateľovi sa automaticky po inštalácii vytvoria súbory potrebné pre správne fungovanie serveru Tang.
Konfigurácia serveru Tang taktiež nie je potrebná.
Po-inštalačný skript zaistí nie len vygenerovanie potrebných súborov, ale aj následnú konfiguráciu služby {\tt tangd}.
Posledný krok používateľa pre aktiváciu služby tangd je reštartovanie služby xinetd.
Finálna verzia balíku Tang čaká na prijatie komunitou a následnú kontrolu jej vývojármi.

Nakoniec, so správnou konfiguráciou serveru Tang pre službu xinetd na našom zariadení OpenWrt, je server Tang schopný obsluhovať klientov bez nutnosti ďalšieho počítača na sieti, s~obmedzeniami, ktoré táto konfigurácia na vstavané zariadenie priniesla.
