Hlavným cieľom tejto práce je naportovať a zdokumentovať tento proces sprístupnenia serveru Tang, na vstavané zariadenia typu WiFi smerovač, s plne modulárnym operačným systémom OpenWrt.
Tým dosiahneme anonymnú správu šifrovacích kľúčov pre domáce siete a siete malých firiem.
Dosiahnutie hlavného cieľa je možné vďaka procesu portovania a krížovej kompilácie na platforme Linux.

Táto práca popisuje problematiku šifrovania a jeho využitie na zabezpečenie pevného disku počítača.
Oboznámuje čitateľa so štruktúrou šifrovaného diskového oddielu podľa LUKS špecifikácie na operačných systémoch typu Linux.
Práca rozoberá možnosti automatizácie odomykania šifrovaných diskov použitím externého servera {\it Key Escrow} alebo {\it Tang}, ktorý vstupuje do procesu ako tretia strana.

Server Key Escrow slúži na ukladanie šifrovacích kľučov.
Na rozdiel od serveru Tang, Key Escrow potrebuje zabezpečenú infraštruktúru a identifikovať každého používateľa ktorý chce jeho služby využívať.
Server Tang poskytuje anonymnú obnovu kľúčov, a teda žiaden z nich nepozná, iba poskytne matematickú operáciu klientovi, ktorý žiada o obnovenie.
Operácie, ktoré server tang vykonáva pri tejto obnove sú detailne opísané v práci.

Krížovú kompiláciu vykonáva upravený kompilátor, ktorý generuje binárny kód spustiteľný na inej platforme, než na ktorej je samotný preklad zdrojových kódov spustený.
Používa sa v prípadoch, keď sú zdrojové kódy spoločné pre viacero ceľových platforiem, na ktorých môže byť program spustený (Linux a Microsoft Windows) alebo rovnaký systém s rôznymi typmi procesorov (16bitový, 32bitový a 64bitový) alebo architektúr (x86\_64, MIPS, ARM).
Často sa využíva na generovanie spustiteľných súborov pre vstavané systémy a pri prekladu na platformy, ktoré niesú shopné kompilácie.

Tento kompilátor je spolu s linkerom a štandardnou knižnicou jazyka C hlavnou súčasťou súboru počítačových programov, nástrojov alebo utilít, nazývaným aj tool-chain.
Tool-chain pre systém OpenWrt je možné vygenerovať použitím buildroot-u z GitHub repozitára openwrt/openwrt.
Tento buildroot obsahuje pomocné skripty a Makefile-y uľahčujúce automatizáciu procesu vytvárania inštalovateľného obrazu systému OpenWrt.
Buildroot je možné nakonfigurovať pre špecifické zariadenie podporované systémom OpenWrt.

Okrem schopnosti vytvárania obrazov systému OpenWrt je tento nástroj používany na vytváranie inštalovateľných softvérových balíkov pre túto platformu.
Pre portnutie, teda vytvorenie balíka, je potrebné vytvoriť adresárovú štruktúru v repozitári feeds.
Táto adresárová štruktúra obsahuje súbory zdrojového kódu pre rôzne softvérové projekty spolu s Makefile-om, ktorý obsahuje direktívy buildroot-u pre spustenie krížovej kompilácie na cielenú architektúru.
Takto vytvorené balíky je možné následne nainštalovať na zariadenia s nainštalovaným systémom OpenWrt.
